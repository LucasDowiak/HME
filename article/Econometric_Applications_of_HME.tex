\documentclass[12pt]{article}

\usepackage{amssymb}
\usepackage{amsfonts}
\usepackage{amsmath}
\usepackage{mathbbol}
\usepackage{threeparttable}
\usepackage{booktabs}
\usepackage{geometry}
\usepackage{pdflscape}
%\usepackage{algorithmic}
%\usepackage{algorithm}
\usepackage{graphicx}

% The hessian:
\newcommand{\bw}[1]{\boldsymbol{\omega}^{#1}}
\newcommand{\Ht}[1]{\mathbf{H}_{t}(#1)}
\newcommand{\HH}[1]{\boldsymbol{H}(#1)}
\newcommand{\HI}[1]{\boldsymbol{H}^{-1}(#1)}

\usepackage[
  bibencoding=auto,
  backend=biber,
  url=true,
  doi=true,
  style=authoryear,
  uniquename=false,
  natbib
]{biblatex}

\addbibresource{hme_references.bib}

\newcommand{\mean}[1]{\bar{#1}}
\newcommand{\gateprod}[2]{\pi_{#1 \longrightarrow #2}}
\newcommand{\sumgateprod}[3]{\pi_{#1 \overset{#3}{\longleftrightarrow} #2}}
\newcommand{\shortsum}[1]{\sum \nolimits_{#1}}
\newcommand{\expmixwt}[0]{\mathbb{\Pi}}
\newcommand{\h}[2]{h^{#1}_{#2}}

\newcommand{\ti}[1]{\textit{#1}}
\newcommand{\iu}[1]{\underline{\textit{#1}}}
\newcommand{\ibu}[1]{\textbf{\underline{\textit{#1}}}}
\newcommand{\T}{\rule{0pt}{2.5ex}}       % Top strut
\newcommand{\Eym}{\mathbb{E}^{m} \left[ y_{t} \right]}
\newcommand{\FnOmegaNaught}[2]{\Omega^{(0)}_{t}( #1, #2 )}
\newcommand{\FnOmegaOne}[4]{\Omega^{(1)}_{t}(#1, #2)(#3, #4)}

\DeclareMathOperator*{\argmax}{arg\,max}
\DeclareMathOperator*{\argmin}{arg\,min}


\setlength{\parindent}{0pt}

\bibliography{hme_references}

\graphicspath{{/Users/lucasdowiak/Git/hme/article/images/}}

 
\title{Econometric Applications of Hierarchical Mixture of Experts}
\author{Lucas C. Dowiak}

\begin{document}
 
\maketitle{}


PhD Program in Economics, City University of New York\smallskip, Graduate Center,

New York, NY, 10016, \textit{Email: ldowiak@gradcenter.cuny.edu}

\qquad

\begin{abstract}
In this article, a novel mixture model is studied.
Named the hierarchical mixture of experts (HME) in the machine learning literature,
the mixture model utilizes a set of covariates and a tree-based architecture to
efficiently allocate each observation to the appropriate local regression.
The nature of the conditional weighting scheme provides the researcher a natural
interpretation of how the local (and latent) sub-populations are formed. The model is
demonstrated by estimating a Mincer earning function using census data. Marginal effects,
robust standard errors, a tree-growing algorithm, and a modest extension are also
discussed.

ˇ\end{abstract}

\vspace{1pt}

\begin{quotation}
\textbf{Keywords}: Hierarchical mixture of experts, expectation maximization

\textbf{JEL Classification}: 
\end{quotation}

\vspace{1pt}

\section{Introduction}

The concepts of mixture models and mixture distributions are old hat
in the economics field. \citet{Hamilton1989} and \citet{GoldfeldQuant1973}
are a few of the pioneering works for time series and cross sectional
regression, respectively. Today, the modern computing environment is dominated by machine
learning, and its reigning champion, the artificial neural network, has
been successfully adapted and studied in the context of applied econometrics.
This essay adds to the small body of literature that employs a novel
neural network architecture to model the weights of a mixture model. In doing
so, the model leverages the highly flexible nature of a neural network but maintain
interpretability and the means to quantify marginal effects.
The model under investigation is called the Hierarchical Mixture
of Experts (HME), a class of mixture models whose defining feature is 
its conditional weighting scheme. The model's origin story traces back to \citet{JJNH1991}.
The authors use a single multinomial classifier to assign, in a probabilistic
sense, input patterns to \textit{local experts}. These experts are
almost always some flavor of regression or classification model.
The multinomial structure that assigns inputs to experts is
referred to as the \textit{gating network}. The authors employ this
mixture of experts (ME) framework to model vowel discrimination in a speech
recognization context. Shortly after, \citet{JordanJacobs1992}
generalize this single-layer multinomial gating network to one with an arbitrary
number of layers. \citet{JordanJacobs1993} then demonstrate
an Expectation-Maximization approach to model estimation that is capable of handling the additional complexity
the generalization requires during optimization. The result
of this extension is a gating network that takes on a tree-like structure,
stemming from an initial multinomial split and filtering down through additional
multinomial partitions of the input space. HME models nest ME models as special case.
Pushing a little further, one additional case is studied as well. As the depth of an HME
grows, so too must the number of experts. In the case of a symmetric HME network,
this growth is geometric with respect to the network's depth. With this in mind,
a further extension can be considered where each expert is not unique, but a member of a fixed
set of experts. This additional model is referred to as a Hierarchical Mixture of Repeated
Experts (HMRE). Figure (\ref{fig:network_comparison}) provides an example
of each of the variations of this class of model.

\bigskip

This essay investigates the adoption of ME and HME models to an applied
econometric framework, with particular attention focused on interpretation of
the gating network and robust inference of parameter estimates. The outline for the
rest of this essay is as follows: the remainder of this section
provides a brief literature review and Section \ref{sec:Model} describes
the model in formal detail. Section \ref{sec:Estimation} discusses the
expectation-maximization approach to estimation while Section \ref{sec:Inference}
concerns itself with robust inference of the estimated parameters.
Section \ref{sec:MarginalEffects} provides detail on how to derive the
marginal effects of the model's covariates. In Section \ref{sec:SimpleExample},
a vary simple demonstration of the HME in action is presented with 
a more economically relevant example of applying the HME model to
a Mincer wage equation in Section \ref{sec:MincerWageEx}. Section
\ref{sec:Conclusion} concludes.

\begin{figure}[!ht]
  \centering
  \includegraphics[width=\textwidth]{network_types.jpeg}
  \caption{Networks \textbf{A} - \textbf{D} depict various network
  architectures that are discussed in this article. For all four networks,
  gating nodes are represented as blue circles and experts as orange
  rectangles. Network \textbf{A} illustrates the original Mixture of
  Experts (ME) architecture with a single multinomial split leading
  to a set of experts one layer down. Networks \textbf{B} and \textbf{C} both
  represent different flavors of a Hierarchical Mixture of Experts (HME). Network
  \textbf{B} is a symmetric network of depth 2 with successive binary splits.
  Network \textbf{C} is an asymmetric network of depth 3 with successive
  binary splits. Network \textbf{D} is an example of the Hierarchical
  Mixture of Repeated Experts (HMRE) architecture. Notice that multiple paths
  exist from the root node $0$ to each expert. Compare this to networks
  \textbf{A} - \textbf{C}, where there is only one unique path from the root
  node to each expert.}
  \label{fig:network_comparison}
\end{figure}


\subsection{Relevant Literature}

ME and HME frameworks have been utilized for both time series and
cross-sectional analysis. Within the cross-sectional literature,
\citet{WaterhouseRobinson1995} puts forth a method to grow an HME from a
single split from the root node. The authors are influenced by the popular
technique used for classification and regression trees \citep{CART1984} and 
apply it to an HME structure. Once the gating structure to an HME tree has    
been grown, the authors suggest an additional trimming algorithm to prevent overfitting.
\citet{FFW1997} extend the approach of \citet{WaterhouseRobinson1995} by altering their
growing algorithm with a mind to scaling the model to handle thousands of
experts. \citet{JordanXuConverge1995} provide an extended discussion on the convergence of the
model used by \citet{JordanJacobs1993}. The authors also suggest
algorithmic improvements to help with estimation. Continuing the theoretical
discussing, \citet{JiangTanner1999} cover convergence rates of an HME model where experts
are from the exponential family with generalized linear mean functions.
\citet{JiangTanner2000} provide regularity conditions on the HME structure for
for a mixture of general linear models estimated by maximum likelihood to
produce consistent and asymptotically normal estimates of the mean response.
The conditions are validated for poisson, gamma, gaussian, and binomial experts.

\bigskip

Alternatively, \citet{WMS1995} provide a detailed discussion 
examining ME applied in a time series context and provide valuable
insights to avoid overfitting the model to the data, a common problem in
neural network applications. The authors' formulation of the model has
close similarities to other non-linear time series models developed in
the late 1980's and early 1990's. A ME time series model sits between the
markov-switching (MS) model of \citet{Hamilton1989} and the smooth transition
auto-regressive (STAR) model of \citet{Terasvirta1994}, borrowing a bit from both.
From an estimation perspective, the ME time series follows close to the
markov-switching model due to the fact that they are both mixture distribution where
each (conditional) distribution represents a different "state" of nature. 
The STAR model, on the other hand, posits only a single distribution and 
different "states" are represented by unique parameter vectors, and 
as the name implies, the parameters transition smoothly
from one state to another over time. The association between the ME, MS, and
STAR models is inverted when it comes to how to frame the time evolution of the
states. From this perspective, the ME model is very similar to a STAR model in
that it also uses the logistic (or multinomial) function to force the probability
of belonging to one state to change over time. For MS models, a discrete state
markov process is used to model this dynamic change over the time dimension.
\citet{HuertaJiangTanner2003} extend \citep{WMS1995} to an HME
framework. Using five and a half decades of monthly US industrial production
data, the authors allow the series to choose between two models, one modeled
as a random walk and the other as trend stationary. In addition, they present a
Bayesian approach to estimation. \citet{CarvalhoTanner2003} lay out the necessary regularity conditions to perform
hypothesis tests on stationary ME time series of generalized linear models (ME-GLM)
using Wald tests. The dual cases of a well-specified and a miss-specified model
are considered. The authors restrict their analysis to ME-GLM models involving
lagged dependent and lagged external covariate variables only. Generalization to
include lagged conditional mean values is left for another time.
\citet{CarvalhoTanner2005} take a similar approach to \citet{CarvalhoTanner2003} but
apply their analysis to a purely auto-regressive context restricted to Gaussian models.
The authors extend arguments in \citet{CarvalhoTanner2003} to non-stationary series and
provide simulated evidence that the BIC is a helpful statistic for selecting the 
appropriate number of experts to include. \citet{CarvalhoTanner2006} re-focus the discussion on ME of time series
regressions restricted to exponential family distributions. Distilling
the available literature at the time, the authors cover the important
topics of estimation and asymptotic properties in the maximum likelihood
framework, selection of the number of experts, model validation and
fitting. \citet{CarvalhoSkoulakis2010} applies ME of a single time series.
Using stock returns, the authors structure the gating network using lagged
dependent variables and an 'external' covariate capturing a measure of the
trade volume at that time.

\bigskip

In this essay estimation and inference is from a maximum likelihood 
perspective and will remain the primary focus. Estimation of ME and HME
models from a Bayesian has received considerable amount of attention as well.
\citet{Waterhouse1995BayesianMF} provided an initial approach to estimating
a ME by combining gaussian priors on the gating and expert parameters
with gamma hyper-parameter priors in an approximating ensemble to the true
joint density of the model. Optimization of the parameter vector for the approximating density
occurs a block of parameters at a time. \citet{UedaGhahramani2002} improve
on \citet{Waterhouse1995BayesianMF} by optimizing for the appropriate number of
experts in addition to  model parameters. \citet{BishopSvenson2003} find
previous bayesian approaches to estimating an HME lacking. Using variational inference, the authors provide a
complete bayesian estimation approach to the log marginal likelihood. With an eye
to prediction, the authors advocate that their approach makes the HME model
easier to estimate without overfitting.

\bigskip

\section{Model} \label{sec:Model}

To start, the HME is presented as a standard mixture model.
For a given input and output pair $(\boldsymbol{x}_{t}, y_{t})$, each expert
provides a probabilistic model relating input row $\boldsymbol{x}_{t}$ to
output $y_{t}$:

\begin{equation} \label{eq:ConditionalDistribution}
  P^{m}_{t} \equiv P^{m}(y_{t}|\boldsymbol{x}_{t}; \boldsymbol{\beta}^{m}), \quad m = 1,2,...,M
\end{equation}

where $m$ is one of the $M$ component experts in the mixture. The experts
are combined with associated weights into a mixture distribution

\begin{equation} \label{eq:staticmixture}
  P(y_{t} | \boldsymbol{x}_{t}; \, \boldsymbol{\beta}) = \sum_{m=1}^{M} \expmixwt(m|t) P^{m}(y_{t} | \boldsymbol{x}_{t}; \boldsymbol{\beta}^{m})
\end{equation}

Here, $\expmixwt(m|t)$ is the probability that the input unit $t$ belongs
to expert $m$ and has the usual restrictions: $0 \leq \expmixwt(m|t) \leq 1$
for each $m$ and $\sum_{m} \expmixwt(m|t) = 1$. The gating network of the model
applies a particular functional form to model $\expmixwt(m|t)$, which includes a
second set of covariates $\boldsymbol{z}_{t}$ and parameter vector $\boldsymbol{\Omega}$:

\begin{equation} \label{eq:mixture}
  P(y_{t} | \boldsymbol{x}_{t}, \boldsymbol{z}_{t}; \, \boldsymbol{\beta}, \boldsymbol{\Omega}) = \sum_{m=1}^{M} \expmixwt(m | \boldsymbol{z}_{t}; \boldsymbol{\Omega}) P^{m}(y_{t} | \boldsymbol{x}_{t}; \boldsymbol{\beta}^{m})
\end{equation}


\subsection{Gating Network and $\expmixwt(m | \boldsymbol{Z}, \boldsymbol{\Omega})$} \label{subsec:GatingNetwork}

The gating network model is structured as a collection of nodes in a tree
structure that branches out in successive layers. The location of these nodes will
be referred to by their address $a$. The root node resides at the apex of the tree
and has the address $0$. The root node then splits into $J$ different nodes,
one level down the tree. The addresses for these $J$ new nodes are 
$1|0, 2|0, ..., J|0$. This type of naming convention continues as the
rest of network is traversed. At its most general, each gating node can yield an
arbitrary number of splits. While a fully generalized gating network is
conceptually attractive, it presents practical challenges for implementation.
In this paper we address several architectures for the gating network, each
with its own set of structural restrictions on the shape of the network and
the number of splits each gating node can take. For arbitrary gating node at address $a$,
we use a multinomial logistic regression to model the split in direction $i$ to be:

\begin{equation} \label{eq:softmax}
  g^{a,i}_{t} \equiv g^{a,i}_{t}(\boldsymbol{z_{t}}, \boldsymbol{\omega}^{a}) = \frac{\exp(\boldsymbol{z_{t}} \  \bw{a,i})}{\sum^{J}_{j=1} {\exp(\boldsymbol{z_{t}} \ \bw{a, j})}}
\end{equation}

The parameters in equation (\ref{eq:softmax}) are subject to the usual
identification restrictions. For the remainder of this essay, the choice is made
to set $\boldsymbol{\omega}^{a,J} = \boldsymbol{0}$ for every gating node.
It is important to keep track of the product path an input vector travels from
one node to another. If the observation index is suppressed, the product path
from one node (say the root node $0$) to another (say $k|\ldots|j|i$)
can be defined as

\begin{equation} \label{eq:gpath}
  \gateprod{g^{0}}{g^{k|\ldots|j|i|0}} =
    \begin{cases} 
       g^{0, \, i} \, g^{i|0, \, j} \ldots g^{\dots|j|i|0, \, k} & \textrm{if path is feasible} \\
       0 & \textrm{otherwise}
    \end{cases}
\end{equation}

If one of the nodes is an expert, then we can define the mixture weight
of expert $m$ for input pattern $i$ to be the product of the path taken
from the root node to expert $m$:

\begin{equation} \label{eq:gpath2}
  \expmixwt(m | \boldsymbol{Z}, \boldsymbol{\Omega}) = \gateprod{g^{0}}{P^{m}}
\end{equation}

For network architectures with multiple paths from the root node to
the same expert (see bottom right of figure (\ref{fig:network_comparison})),
we can index these multiples paths by $l$ so that

\begin{equation} \label{eq:pathsums}
  \expmixwt(m | \boldsymbol{Z}, \boldsymbol{\Omega}) = \shortsum{l} \sumgateprod{g^{0}}{P^{m}}{l} 
\end{equation}


By collecting and summing all possible paths from the root node to each
expert, the conditional probability given in equation (\ref{eq:mixture}) can be
expanded and expressed as:

\begin{equation} \label{eq:contribution}
  \begin{split}
    P(y_{t}| \boldsymbol{x}_{t}, \boldsymbol{z}_{t}; \boldsymbol{\Omega}, \boldsymbol{\beta}) =& \sum_{m} \expmixwt(m | \boldsymbol{z}_{t}, \boldsymbol{\Omega}) P^{m}(y_{t}| \boldsymbol{x}_{t}; \boldsymbol{\beta}^{m}) \\ 
      =& \sum_{m} \left( \shortsum{l} \sumgateprod{g^{0}}{P^{m}}{l} \right)  P^{m}(y_{t}| \boldsymbol{x}_{t}; \boldsymbol{\beta}^{m})
  \end{split}
\end{equation}

If we concatenate the parameters of the gating network with the parameters of
the experts as $\boldsymbol{\theta} = [\boldsymbol{\beta} \,\, \boldsymbol{\Omega}]$,
then the product of these individual probabilities across the full sample size $T$ yields
the likelihood function. 

\begin{equation} \label{eq:likelihood}
  \mathcal{L}(\boldsymbol{\theta}| \boldsymbol{y}, \boldsymbol{X}, \boldsymbol{Z}) = \prod_{t}  \sum_{m} \left( \shortsum{l} \sumgateprod{g^{0}}{P^{m}}{l} \right)  P^{m}(y_{t}| \boldsymbol{x}_{t}; \boldsymbol{\beta}^{m})
\end{equation}

And taking its log yields the log likelihood

\begin{equation} \label{eq:loglikelihood}
  \boldsymbol{l}(\boldsymbol{\theta}| \boldsymbol{y}, \boldsymbol{X}, \boldsymbol{Z}) = \sum_{t} \log \left[ \sum_{m} \left( \shortsum{l} \sumgateprod{g^{0}}{P^{m}}{l} \right) P^{m}(y_{t}| \boldsymbol{x}_{t}; \boldsymbol{\beta^{m}}) \right]
\end{equation}

The functional form of the log likelihood (\ref{eq:loglikelihood}) does not
lend itself easily to direct optimization, but a well established
technique using expectation maximization \citep{EM_DLR1977} to estimate mixture
models is available. This was the primary insight of \citet{JordanJacobs1993}'s
original paper.


\section{The EM Set-Up} \label{sec:Estimation}

The EM approach to estimating an HME model starts by suggesting that if a
researcher had perfect information, each input vector $\boldsymbol{x}_{t}$ could be matched
to the expert $P^{m}$ that generated it with certainty. If a set of indicator
variables is introduced that captures this certainty, an \textit{augmented}
version of the likelihood in equation (\ref{eq:likelihood}) can be put forward.
Define the indicator set as:

\begin{equation} \label{eq:indicator}
  I_{t}(m) = \begin{cases} 
     1 & \textrm{if observation $t$ is generated by expert $m$} \\
     0 & \textrm{otherwise}
             \end{cases}
\end{equation}

We can then reformulate the likelihood equation

\begin{equation}  \label{eq:auglikelihood}
  \mathcal{L}_{c}(\boldsymbol{\theta}| \boldsymbol{y}, \boldsymbol{X}, \boldsymbol{Z}) = \prod_{t} \prod_{m} \left[ \left( \shortsum{l} \sumgateprod{g^{0}}{P^{m}}{l} \right)  P^{m}(y_{t}| \boldsymbol{x}_{t}; \boldsymbol{\beta^{m}}) \right]^{I_{t}(m)}
\end{equation}

leading to the complete-data log-likelihood

\begin{equation}  \label{eq:augloglikelihood}
  \boldsymbol{l}_{c}(\boldsymbol{\theta}| \boldsymbol{y}, \boldsymbol{X}, \boldsymbol{Z}) = \sum_{t} \sum_{m} I_{t}(m) \left[\log \left( \shortsum{l} \sumgateprod{g^{0}}{P^{m}}{l} \right) + \log P^{m}(y_{t}| \boldsymbol{x}_{t}; \boldsymbol{\beta^{m}}) \right]
\end{equation}

As mentioned previously, summing over multiple paths $l$ in equation
(\ref{eq:augloglikelihood}) is only necessary in the HMRE case. For the ME
and HME cases, $l$ equals 1, simplifying the first log in (\ref{eq:augloglikelihood})
to $\log(\gateprod{g^{0}}{P^{m}})$. Going forward, we will focus our analysis on
the ME and HME specifications with work on the HMRE case left for another time.

\subsection{E-Step} \label{sec:Estep}
The E-step of the algorithm performs an expectation over the complete
log-likelihood equation (\ref{eq:augloglikelihood}), where the expectation
includes the additional information contained in the expert regressions.
One of the results of this expectation is the creation of second set of
weights $h^{a}$ that parallel the weights from the gating network $g^{a}$
discussed in section (\ref{subsec:GatingNetwork}). For an HME model:

\begin{equation} \label{eq:Estep}
  \begin{split}
  Q(\boldsymbol{\theta}) = \mathbb{E} \left [ \boldsymbol{l}_{c}(\boldsymbol{\theta}| \boldsymbol{y}, \boldsymbol{X}, \boldsymbol{Z}) \right] & = \sum_{t}\sum_{m} \mathbb{E} \left[ I_{t}(m) \right] \left[ \log P^{m}(y_{t}| \boldsymbol{x}_{t}; \boldsymbol{\beta}^{m}) + \log \gateprod{g^{0}_{t}}{P^{m}_{t}} \right] \\ 
   & = \sum_{t} \sum_{m} \gateprod{h^{0}_{t}}{P^{m}_{t}} \left[ \log P^{m}(y_{t}| \boldsymbol{x}_{t}; \boldsymbol{\beta}^{m}) + \log \gateprod{g^{0}_{t}}{P^{m}_{t}} \right] \\
   & = \sum_{t} \sum_{m} \gateprod{h^{0}_{t}}{P^{m}_{t}}  \log P^{m}(y_{t}| \boldsymbol{x}_{t}; \boldsymbol{\beta}^{m})   +       \sum_{t} \sum_{m} \gateprod{h^{0}_{t}}{P^{m}_{t}}  \log \gateprod{g^{0}_{t}}{P^{m}_{t}} \\
   & = \sum_{t} Q^{(1)}_{t} (\boldsymbol{\beta}) + \sum_{t} Q^{(2)}_{t} (\boldsymbol{\Omega}) \\
   & = \sum_{t} Q_{t}(\boldsymbol{\theta})
 \end{split}
\end{equation}

Here $\gateprod{h^{0}}{h^{k,\dots|j|i|0}}$ is analogous to equation (\ref{eq:gpath})

\begin{equation} \label{eq:hpath}
  \gateprod{h^{0}}{h^{k|\ldots|j|i|0}} =
    \begin{cases} 
       h^{0, \, i} \ h^{i|0, \, j} \ldots h^{\dots|j|i|0, \, k} & \textrm{if path is feasible} \\
       0 & \textrm{otherwise}
    \end{cases}
\end{equation}

and the $h^{a, i}$ are arrived at using Bayes' theorem.

\begin{equation} \label{eq:posteriornode}
  \h{a,i}{t} = \frac{g^{a, i} \shortsum{k} P^{k}_{t} \gateprod{g^{i|a}_{t}}{P^{k}}}{\shortsum{j} g^{a, j} \shortsum{m} P^{m}_{t} \gateprod{g^{j|a}_{t}}{P^{m}}}
\end{equation}


So, now we have two different forms of weights, $g$'s and $h$'s. The
way the $g$'s are formed in equation (\ref{eq:softmax}), they are only
functions of the nodes in the gating network, separate from the
expert regressions and the information they contain. For this reason,
\citet{JordanJacobs1993} refer to $g$'s as \textit{priors}.
The $h$'s draw from both the gating network and the expert regressions and
are referred to as \textit{posterior} weights.


\subsection{M-Step} \label{sec:Mstep}

One of the more attractive features of using EM to a optimize a
HME is how the log-likelihood function compartmentalizes into a set 
of independent functions which can be individually optimized. After
taking the expectation of the log-likelihood function (\ref{eq:Estep}), the
parameters governing each expert and each gating network can be grouped
together and optimized on their own. For the experts we have:

\begin{equation} \label{eq:BetaUpdate}
  \argmax_{\boldsymbol{\beta^{m}}} \sum_{t} \gateprod{h^{0}_{t}}{P^{m}_{t}} \log P^{m} (y_{t}| \boldsymbol{x}_{t}; \boldsymbol{\beta}^{m})
\end{equation}

and for the gating nodes:

\begin{equation} \label{eq:OmegaUpdate}
  \argmax_{\boldsymbol{\omega^{a}}} \sum_{t} \gateprod{h^{0}_{t}}{h^{a}_{t}} \log g( \boldsymbol{z}_{t}, \boldsymbol{\omega}^{a})
\end{equation}

It is worth noting that the weights in these optimizations 
$\gateprod{h^{0}_{t}}{h^{a}_{t}}$ are provided to the M-step by the
E-step and should be considered constant values.

\subsection{The EM-Algorithm}

The EM algorithm iterates back-and-forth between the E-step and the M-step. 
Given the data ($\boldsymbol{y}_{t}$, 
$\boldsymbol{X}_{t}$, $\boldsymbol{Z}_{t}$) and the current set of parameters
$\boldsymbol{\theta}^{k}$, the expected value of the complete log-likelihood
(eq. (\ref{eq:augloglikelihood})) is found, resulting in the deterministic
function Q($\boldsymbol{\theta}^{k}$). In essence,
the main objective of the E-step is to derive the values of the posterior weights
($\h{a,i}{t}$) using equations (\ref{eq:ConditionalDistribution}), (\ref{eq:softmax}), (\ref{eq:gpath}), (\ref{eq:hpath}) and (\ref{eq:posteriornode}).
Once the posterior weights have been calculated in the E-step,
the M-step holds them constant and then re-estimates the parameter vector:

\begin{equation}
    \boldsymbol{\theta}^{k + 1}  = \argmax_{\boldsymbol{\theta}} Q(\hat{\boldsymbol{\theta}}^{k}) = \left[ \argmax_{\boldsymbol{\beta}} Q^{(1)} ( \hat{\boldsymbol{\beta}}^{k} ) \,\,\,\, \argmax_{\boldsymbol{\Omega}} Q^{(2)} ( \hat{\boldsymbol{\Omega}}^{k} ) \right]
\end{equation}

Again, due to the separable nature of $Q$ (see the middle equality of eq (\ref{eq:Estep})),
the parameters of each expert regression and each gating node can be
updated one-at-a-time with equations (\ref{eq:BetaUpdate}) and
(\ref{eq:OmegaUpdate}), respectively. This separability of the Q function when
applied to finite mixture problems was noticed in the original and excellent
work of \citet{EM_DLR1977}. See Section 4.3 for the authors' example. 

\bigskip

It is worth mentioning a few more of the remarkable properties of the EM algorithm
that are also established in \citet{EM_DLR1977}:

\begin{enumerate}
  \item Given a sequence of parameter values produced by the General EM algorithm,
  $\boldsymbol{\theta}^{k} \rightarrow \boldsymbol{\theta}^{k+1} \rightarrow \ldots \rightarrow \boldsymbol{\theta}^{k+n}$,
  the sequence of values are non-decreasing in their log-likelihood values
  $\boldsymbol{l}(\boldsymbol{\theta}^{k}| \cdot) \leq \boldsymbol{l}(\boldsymbol{\theta}^{k+1}| \cdot) \leq \ldots \leq \boldsymbol{l}(\boldsymbol{\theta}^{k+n}| \cdot)$
  and are strictly increasing in the Q function $Q(\boldsymbol{\theta}^{k}) < Q(\boldsymbol{\theta}^{k+1}) < \ldots < Q(\boldsymbol{\theta}^{k+n})$.

  \item The sequence of parameter values produced by the General EM algorithm
  converges to a fixed point such that in the limit:
    \begin{equation}
      \boldsymbol{\theta}^{*}  = \argmax_{\boldsymbol{\theta}} Q(\boldsymbol{\theta}^{*})
    \end{equation}
  Crucially, the vector that the general EM algorithm converges
  to is a maximum likelihood estimator of the original log-likelihood equation
  defined in (\ref{eq:loglikelihood}). That is, 
  $\boldsymbol{l}(\boldsymbol{\theta}^{*}| \boldsymbol{y}, \boldsymbol{X}, \boldsymbol{Z}) \geq \boldsymbol{l}(\boldsymbol{\theta}| \boldsymbol{y}, \boldsymbol{X}, \boldsymbol{Z})$
  for all $\boldsymbol{\theta} \in \boldsymbol{\Theta}$.
\end{enumerate}

\section{Inference} \label{sec:Inference}

When considering inference, it is worth thinking about what would motivate
a researcher to turn to an HME model in the first place. At times,
a researcher may suspect that a latent structure exists within the data
and that a single regression $y_{t} = \boldsymbol{x}_{t} \boldsymbol{\beta}$
may mask a critical change in relationship depending on membership to
some unknown sub-group $j$ of the data $y_{tj} = \boldsymbol{x}_{t} \boldsymbol{\beta}_{j}$.
A wide variety of time series, especially those with longer histories,
experience changes in behavior over time. They can be subjected to sharp one-off
structural breaks or the changes can be more gradual changes over time. Regardless of
the context, any latent structural change in the data generating
process may also introduce some hidden form of heterogeneity to the error terms. 
Rather than taking a firm stance on any concealed structure, an HME
setup ideally limits the work the researcher needs to do to specifying
a set of well-chosen conditioning variables $\boldsymbol{Z}$ to feed through the gating
network. This limited workload may come at a cost, though. By allowing the 
gating network to find its own mixture allocations, the odds of arriving at a
misspecified model becomes a concern. To guard against this, we use a sandwich
estimator for the variance-covariance matrix:

\begin{equation} \label{eq:robustgatevarcov}
  \boldsymbol{V}(\boldsymbol{\theta}) = \HI{\boldsymbol{\theta}} \boldsymbol{G}(\boldsymbol{\theta}) \HI{\boldsymbol{\theta}}
\end{equation}



where $\boldsymbol{G}(\boldsymbol{\theta})$ is the sum of the outer products of the score 
vectors

\begin{equation}
  \boldsymbol{G}(\boldsymbol{\theta}) = \sum_{t} \boldsymbol{S}_{t}(\boldsymbol{\theta}) \boldsymbol{S}_{t}(\boldsymbol{\theta})^\top
\end{equation}

and $\HH{\boldsymbol{\theta}}$ is the empirical Hessian:

\begin{equation}
  \HH{\boldsymbol{\theta}} = \frac{1}{T} \sum_{t} \Ht{\boldsymbol{\theta}}
\end{equation}

The following sections discusses how to form the score and hessian matrices 
for the log-likelihood described in equation (\ref{eq:loglikelihood}).

\subsection{The Score} \label{sec:TheScore}

The notation is tedious but the acyclic nature of the
gating network makes interpretation of the score vectors clear and
straightforward. The full score vector is the concatenated scores  
of each gating node and those of each local expert.

\begin{equation}
  \boldsymbol{S}_{t}(\boldsymbol{\theta}) = [ \boldsymbol{S}_{t}(\boldsymbol{\beta})^{\top} \,\, \boldsymbol{S}_{t}(\boldsymbol{\Omega})^{\top} ]^{\top}
\end{equation}

Starting with
parameters of the gating network, they can be partitioned 
in some logical order into the sub-vectors of each node's individual
score.

\begin{equation}
  \boldsymbol{S}_{t}(\boldsymbol{\Omega}) = [ \boldsymbol{S}_{t}(\boldsymbol{\omega}^{0})^{\top} \,\, \boldsymbol{S}_{t}(\boldsymbol{\omega}^{1 | 0})^{\top} \,\, \boldsymbol{S}_{t}(\boldsymbol{\omega}^{2 | 0})^{\top} \,\, \ldots \, ]^{\top}
\end{equation}

\begin{equation}
  \boldsymbol{S}_{t}(\boldsymbol{\omega}^{a}) = [ \boldsymbol{S}_{t}(\bw{a, 1})^{\top} \,\, \ldots \,\, \boldsymbol{S}_{t}(\bw{a, J - 1})^{\top} ]^{\top}
\end{equation}

In what follows, the functions $m(a)$ and $m(a, i)$ will be used to return
a subset of experts from a general HME model. The function $m(a)$ will
return the set of all experts that are ancestors of node $a$, while
$m(a, i)$ returns the set of experts that are ancestors from branch
$i$ of node $a$. For instance, in network $\boldsymbol{C}$ of Figure
\ref{fig:network_comparison}, $m('1|0') = \{'1|1|1|0', \, '2|1|1|0', \, '2|1|0'\}$,
$m('1|0', 1) = \{'1|1|1|0', \, '2|1|1|0'\}$, and $m('1|0', 2) = \{'2|1|0'\}$.
For a generic gating node $a$ we can define the individual score for 
sample $t$ as:

\begin{equation} \label{eq:gateScore}
  \boldsymbol{S}_{t}(\bw{a, i}) = \frac{\partial \boldsymbol{l}_{t}(\boldsymbol{\theta}^{*}| \boldsymbol{y}, \boldsymbol{X}, \boldsymbol{Z}) }{\partial \boldsymbol{\omega}^{a,i}} = \left[ \frac{ \FnOmegaNaught{a}{i} }{ \sum_{m} \gateprod{g^{0}_{t}}{P^{m} } P^{m}_{t} } \right] \boldsymbol{z}_{t}^{\top} 
\end{equation}

with

\begin{equation} \label{eq:omegaScore}
  \FnOmegaNaught{a}{i} = \left( (1 - g^{a, i}_{t}) \sum_{k}^{m(a, i)} \gateprod{g^{0}_{t}}{P^{k}} P^{k}_{t} - \sum_{j \neq i} g^{a,j}_{t} \sum_{k'}^{m(a, j)} \gateprod{g^{0}_{t}}{P^{k'}}  P^{k'} \right)
\end{equation}

In expression (\ref{eq:gateScore}) above, $ \FnOmegaNaught{a}{i} \left[ \sum_{m} \gateprod{g^{0}_{t}}{P^{m}} \right]^{-1}$
is the instantaneous rate of change of the $t$'th contribution to
the log-likelihood caused by a small pertubation of $\bw{a, i}$.
At the maximium likelihood estimator $ \theta^{*}$, the sum
of (\ref{eq:omegaScore}) should be approximately zero.
This implies that over the sample T, the optimal $\bw{a, i}$
balances any gain of moving more weight to the set of experts that can be
reached by taking direction $i$ at node $a$ against the loss suffered by 
removing weight from the experts at the end of any path $j$ that does not 
equal $i$.

\bigskip

Turning our attention to the expert regressions, the exact functional form
of the score vector depends on the type of regression we wish to run. In most
cases, all experts in an HME model are from the same family
(\citet{HuertaJiangTanner2003} is a notable exception). When all experts share
the same functional form, it is standard to accept the restriction that no experts
in the HME model produce the same parameter vector
$\boldsymbol{\beta}^{j} \neq \boldsymbol{\beta}^{k}$. Such an HME is defined by
\citet{JiangTanner2000} as being \textit{irreducible}. The irreducibility of an HME
plays a critical role in guaranteeing the convergence of the model.
In this essay, each HME discussed will employ a set of experts running 
a standard linear regression model with Gaussian errors. To aid with model optimization,
the specification of the parameter vector for each regression,
 $\boldsymbol{\beta}^{m} = [\beta_{0}^{m} \,\, \ldots \,\, \beta_{k}^m \,\, \phi^{m}]^{\top}$,
takes on a unique form where we model the log variance explicitly:
$\phi = \log \sigma^{2}$.

\begin{equation}
  P^{m}(y_{t} | \boldsymbol{x}_{t}; \, \boldsymbol{\beta}^{m}, \phi^{m}) = \left( 2 \pi \exp ( \phi^{m} ) \right)^{-\frac{1}{2}} \exp{ \left( -\frac{  ( y_{t} - \boldsymbol{x}_{t} \boldsymbol{\beta}^{m} )^{2}  }{2 \exp (\phi^{m}) } \right) }
\end{equation}

To help save space in the sections below, the following shorthand notation will be used 
to denote the residual of each local expert: $\epsilon^{m}_{t} = y_{t} - \boldsymbol{x}_{t} \boldsymbol{\beta}^{m}$.
In this case the score vector for all expert regressions can be expressed as:

\begin{equation}
  \boldsymbol{S}_{t}(\boldsymbol{\beta}) = [ \boldsymbol{S}_{t}(\boldsymbol{\beta}^{1})^{\top} \,\, \ldots \,\, \boldsymbol{S}_{t}(\boldsymbol{\beta}^{M})^{\top} ]^{\top}
\end{equation}

\begin{equation} \label{eq:expertScore}
  \boldsymbol{S}_{t}(\boldsymbol{\beta^{m}}) = \left[ \frac{\partial \boldsymbol{l}_{t}}{\partial \boldsymbol{\beta}^{m}}^{\top} \,\,\,\, \frac{\partial \boldsymbol{l}_{t}}{\partial \phi^{m}} \right]^{\top}
\end{equation}

with

\begin{equation} \label{eq:ExpertScoreBeta}
  \frac{\partial \boldsymbol{l}_{t}}{\partial \boldsymbol{\beta}^{m}} = \gateprod{g^{0}_{t}}{P^{m}} \frac{ \epsilon^{m}_{t} }{\exp{(\phi^{m})}} \boldsymbol{x}_{t}^{\top}
\end{equation}
 
and

\begin{equation} \label{eq:ExpertScoreVariance}
  \frac{\partial \boldsymbol{l}_{t}}{\partial \phi^{m}} = \frac{1}{2} \gateprod{g^{0}_{t}}{P^{m}} \left( \frac{( \epsilon^{m}_{t} )^{2}}{\exp{(\phi^{m})}} - 1 \right)
\end{equation}

Expressions (\ref{eq:ExpertScoreBeta}) and (\ref{eq:ExpertScoreVariance})
are the same as the score vectors for a single OLS regression but with
the associated prior weights added on.


\subsection{The Hessian} \label{sec:TheHessian}

The hessian, admittedly, has a complicated form. At its most general it
can be written as $\Ht{\boldsymbol{\theta}}$ in the equation below. The
exact nature of the full hessian depends critically on the structure of the
gating network and the locations of the gate and expert nodes in relation to 
each other.

\begin{equation} \label{eq:FullHessian}
  \Ht{\boldsymbol{\theta}} =  \begin{bmatrix}
  \Ht{\boldsymbol{\beta}^{1}}                                               & \boldsymbol{0}                                                           & \cdots & \Ht{ \boldsymbol{\beta}^{1}, \,  \bw{0, 1} }                & \cdots & \Ht{ \boldsymbol{\beta}^{1}, \,  \bw{\boldsymbol{\cdot}, J - 1} }  \\
  \boldsymbol{0}                                                            & \Ht{\boldsymbol{\beta}^{2}}                                              & \cdots & \Ht{\boldsymbol{\beta}^{2}, \, \bw{0, 1}}                   & \cdots & \Ht{\boldsymbol{\beta}^{2}, \, \bw{\boldsymbol{\cdot}, J - 1}}     \\
  \vdots                                                                    & \vdots                                                                   & \ddots & \vdots                                                      &        & \vdots                                                             \\
  \Ht{ \boldsymbol{\beta}^{1}, \,  \bw{0, 1} }^{\top}                       & \Ht{ \boldsymbol{\beta}^{2}, \,  \bw{0, 1} }^{\top}                      & \cdots & \Ht{\bw{0, 1}}                                              & \cdots & \Ht{\bw{0, 1}, \, \bw{\boldsymbol{\cdot}, J - 1} }                 \\
  \vdots                                                                    & \vdots                                                                   &        & \vdots                                                      & \ddots & \vdots                                                             \\
  \Ht{ \boldsymbol{\beta}^{1}, \,  \bw{\boldsymbol{\cdot}, J - 1} }^{\top}  & \Ht{ \boldsymbol{\beta}^{2}, \,  \bw{\boldsymbol{\cdot}, J - 1} }^{\top} & \cdots & \Ht{ \bw{0, 1}, \,  \bw{\boldsymbol{\cdot}, J - 1} }^{\top} & \cdots & \Ht{ \bw{\boldsymbol{\cdot}, J - 1}}                               \\
    \end{bmatrix}
\end{equation}

Before getting into the details of matrix (\ref{eq:FullHessian}), it will
be worth noting the parameter combinations that will result in  zero-valued
elements of $\boldsymbol{H}_{t}$. First, there are no second-order
partial derivatives across experts, meaning that for any experts
$m$, $k$ where $m \neq k$, then
$\frac{\partial \boldsymbol{l}^{2}_{t}}{\partial \boldsymbol{\beta}^{m} \partial \boldsymbol{\beta}^k} = \boldsymbol{0}$.
This can already be seen in equation (\ref{eq:FullHessian}). Not noted in
(\ref{eq:FullHessian}) are two additional cases where elements of $\boldsymbol{H}_{t}$ will
equal zero. One, if the ancestors of two gating nodes have no shared experts. That is,
for nodes $a$ and $b$, if $m(a) \cap m(b) = \{\}$, then 
$ \frac{\partial^{2} \boldsymbol{l}_{t}}{\partial \boldsymbol{\omega}^{a, i} \partial \boldsymbol{\omega}^{b,n}} = 0$.
Two, if the gating path to expert $m$ does not go through node $a$ then
there are also no second-order partial derivatives. That is, if $g^{a, \cdot}$ is not a
component of $\gateprod{g^{0}_{t}}{P^{m}}$, then
$\frac{\partial \boldsymbol{l}^{2}_{t}}{\partial \boldsymbol{\beta}^{m} \partial \bw{a, i}} = \boldsymbol{0}$.

\bigskip

Also, the expressions for the cross-partial derivatives
between a gating parameter vector and an expert parameter vector can differ
based on the relative position between $\bw{a, i}$ and
$\boldsymbol{\beta}^{m}$ in the HME structure.
For instance, start at the root node and consider what path is needed to
traverse the network to expert $m$. When arriving at
node $a$, if the direction needed to take to reach expert $m$ is along
branch $i$, then $\bw{a,i}$ will be called an \textit{explicit} parameter
vector with respect to expert $m$. If taking branch $i$ leads to a different expert
than $m$, then $\bw{a,i}$ will be referred to as an \textit{implicit}
parameter vector. Now, define $\mathbb{1}\{a, i, m\}$ as an indicator function
that equals one if $\bw{a, i}$ is an explicit parameter vector to expert $m$
and zero if it is an implicit parameter vector (it can only be one or the other).
With this notation, the details to the hessian in equation (\ref{eq:FullHessian})
can now be tackled. Starting with equation (\ref{eq:gateScore}), the
second-order partial derivatives for a pair of gating vectors is:

\begin{equation} \label{eq:nodehessian}
  \frac{\partial^{2} \boldsymbol{l}_{t}}{\partial \boldsymbol{\omega}^{a, i} \partial \boldsymbol{\omega}^{b,n}} = \left[ \sum_{m} \gateprod{g^{0}_{t}}{P^{m}} \right]^{-2} \FnOmegaNaught{a}{i} \cdot \FnOmegaNaught{b}{n} \, \boldsymbol{z}^{\top}_{t} \boldsymbol{z}_{t} \, + \, \left[ \sum_{m} \gateprod{g^{0}_{t}}{P^{m}} \right]^{-1} \boldsymbol{z}^{\top}_{t} \, \frac{ \partial \FnOmegaNaught{a}{i} }{\partial \bw{b, n}}^{\top}
\end{equation}

where

\begin{equation}
  \frac{ \partial \FnOmegaNaught{a}{i} }{\partial \boldsymbol{\omega}^{b, n}} = \sum_{k}^{m(a) \, \cap \, m(b) } \sum_{j}^{J - 1} \sum_{l}^{J - 1} \left\{ \left( \mathbb{1}\{a, j, k\} - g^{a,j}_{t} \right) \left( \mathbb{1}\{b, l, k\} - g^{b,l}_{t} \right) \boldsymbol{\cdot} \FnOmegaOne{a}{i}{b}{l} \boldsymbol{\cdot} P^{k} \right\}  \boldsymbol{z}^{\top}_{t}
\end{equation}

and

\begin{equation} \label{eq:ExpertIntersection}
  \FnOmegaOne{a}{i}{b}{n} = \begin{cases} 
    \sum_{m}^{m(a,i) \, \cap \, m(b, n)}  \gateprod{g^{0}_{t}}{P^{m}}  & \textrm{if $m(a,i) \cap m(b, n) \neq \{ \}$} \\
     0 & \textrm{if $m(a,i) \cap m(b, n) = \{ \}$}
             \end{cases}
\end{equation}

Starting from equations (\ref{eq:ExpertScoreBeta}) and (\ref{eq:ExpertScoreVariance}),
the next set of equations express the second-order partial derivatives for parameters
of each individual expert:

\begin{align}
  \frac{\partial \boldsymbol{l}^{2}_{t}}{\partial (\boldsymbol{\beta}^{m})^{2}} &= \frac{ \gateprod{g^{0}_{t}}{P^{m}} }{ \exp{(\phi^{m})} } \boldsymbol{x}_{t}^{\top} \boldsymbol{x}_{t} \\
  \frac{\partial \boldsymbol{l}^{2}_{t}}{\partial \boldsymbol{\beta}^{m} \partial \phi^{m}} &= - \gateprod{g^{0}_{t}}{P^{m}} \frac{ \epsilon^{m}_{t} }{\exp{(\phi^{m})}} \boldsymbol{x}_{t}^{\top} \\
  \frac{\partial \boldsymbol{l}^{2}_{t}}{\partial (\phi^{m})^{2}} &= \frac{ \gateprod{g^{0}_{t}}{P^{m}} (\epsilon^{m}_{t})^{2} }{ 2 \exp{(\phi^{m})} }
\end{align}

And to round it out, the expression for the cross-partial derivatives
between a gating parameter vector and an expert parameter vector can 
be expressed as:

\begin{align}
  \frac{\partial \boldsymbol{l}^{2}_{t}}{\partial \boldsymbol{\beta}^{m} \partial \bw{a, i}} &= \gateprod{g^a_{t}}{P^{m}} \left( \mathbb{1}\{a, i, m\} - g^{a,i}_{t} \right) \frac{ \epsilon^{m} }{\exp{(\phi^{m})}} \boldsymbol{x}_{t}^{\top} \boldsymbol{z}_{t} \\
  \frac{\partial \boldsymbol{l}^{2}_{t}}{\partial \phi^{m} \partial \bw{a, i}} &=  \frac{1}{2} \gateprod{g^{0}_{t}}{P^{m}} \left( \mathbb{1}\{a, i, m\} - g^{a,i}_{t} \right) \left( \frac{( \epsilon^{m} )^{2}}{\exp{(\phi^{m})}} - 1 \right) \boldsymbol{z}_{t}^{\top} \label{eq:ExpertNodeHessian}
\end{align}

Given the collection of equations (\ref{eq:nodehessian}) - (\ref{eq:ExpertNodeHessian}),
the individual elements of the hessian in equation (\ref{eq:FullHessian})
can be expressed as:

\begin{align}
  \Ht{\boldsymbol{\beta}^{m}} &= \begin{bmatrix}
    \frac{\partial \boldsymbol{l}^{2}_{t}}{\partial (\boldsymbol{\beta}^{m})^{2}}  &  \frac{\partial \boldsymbol{l}^{2}_{t}}{\partial \boldsymbol{\beta}^{m} \partial \phi^{m}}     \\
    \frac{\partial \boldsymbol{l}^{2}_{t}}{\partial \boldsymbol{\beta}^{m} \partial \phi^{m}}^{\top}       &  \frac{1}{2} (\epsilon^{m}_{t})^{2}
    \end{bmatrix} \\
    \Ht{\boldsymbol{\beta}^{m}, \, \bw{a, i}} &= \begin{bmatrix}
      \frac{\partial \boldsymbol{l}^{2}_{t}}{\partial \boldsymbol{\beta}^{m} \partial \bw{a, i}} & \frac{\partial \boldsymbol{l}^{2}_{t}}{ \partial \phi^{m} \partial \bw{a, i} } \\
      \frac{\partial \boldsymbol{l}^{2}_{t}}{\partial \phi^{m} \partial \bw{a, i}}^{\top}        & \frac{\partial \boldsymbol{l}^{2}_{t}}{\partial (\phi^{m})^{2}}
    \end{bmatrix} \\
    \Ht{\bw{a, i}, \, \bw{b, n}} &= \frac{\partial^{2} \boldsymbol{l}_{t}}{\partial \boldsymbol{\omega}^{a, i} \partial \boldsymbol{\omega}^{b,n}}
\end{align}

\section{Marginal Effects} \label{sec:MarginalEffects}

Due to the complexity of the model's structure and the ability to 
place covariates in either the gating network, the expert regressions,
or both, viewing the relationship between the covariates and the dependent
variable through their marginal effects may provide a simplifying lens of the 
model's governing principles. Just as for logistic and multinomial regression,
the marginal effects of an HME model have a closed form solution. Starting
with equation (\ref{eq:mixture}) we replace the expert distributions
$P^{m}_{t}$ with the expected output for each of the $m$ regressions and use
the relationship in equation (\ref{eq:gpath2}) to arrive at: 

\begin{equation} \label{eq:mixture2}
  \mathbb{E} \left[ y_{t} | \boldsymbol{x}_{t}, \boldsymbol{z}_{t}, \boldsymbol{\theta} \right] = \sum_{m=1}^{M} \gateprod{g^{0}_{t}}{m_{t}} \mathbb{E} \left[ y_{t} | \boldsymbol{x}_{t}, \boldsymbol{z}_{t}, \boldsymbol{\theta}, m \right]
\end{equation}

In what follows, $\mathbb{E} \left[y_{t}\right]$ and $\Eym$ will be used as shorthand
for $\mathbb{E} \left[ y_{t} | \boldsymbol{x}_{t}, \boldsymbol{z}_{t}, \boldsymbol{\theta} \right]$
and $\mathbb{E} \left[ y_{t} | \boldsymbol{x}_{t}, \boldsymbol{z}_{t}, \boldsymbol{\theta}, m \right]$,
respectively. The functional form of the marginal effect depends on where the variables
appear in the model. Our existing notation labels the covariates in gating
network as $\boldsymbol{Z}$ and the covariates in the expert regressions 
as $\boldsymbol{X}$. As seen later, the variables belonging to
$\boldsymbol{Z}$ and $\boldsymbol{X}$ do not need to be mutually
exclusive. There is also no requirement that they differ at all.
In light of this, a few more notational definitions are needed to cover
all the cases:

\begin{itemize}  
  \item $\boldsymbol{T} = \boldsymbol{Z} \cup \boldsymbol{X}$
  \item $\boldsymbol{V} = \boldsymbol{Z} \cap \boldsymbol{X}$
  \item $\boldsymbol{U}_{Z} = \boldsymbol{Z} \setminus \boldsymbol{X}$
  \item $\boldsymbol{U}_{X} = \boldsymbol{X} \setminus \boldsymbol{Z}$
\end{itemize}


The full list of variables considered in the model is labeled $\boldsymbol{T}$.
Covariates that appear in both the gating network and the expert regressions
are collect in $\boldsymbol{V}$. $\boldsymbol{U}_{Z}$ and $\boldsymbol{U}_{X}$ are
used to label variables that appear only in the gating network or only in the expert
regressions, respectively. With this notation, we can express the full marginal effects
of the HME by where the explanatory variables appear in the model.

\begin{equation} \label{eq:hme_marginal_effect_def}
  \frac{\partial \mathbb{E} \left[y_{t}\right]}{\partial \boldsymbol{T}} \equiv \boldsymbol{\Delta}_{t} = \sum_{m=1}^{M} \boldsymbol{\Delta}^{m}_{t} = \sum_{m=1}^{M} \left[ \frac{\partial \Eym}{\partial \boldsymbol{U}_{Z}}   \quad   \frac{\partial \Eym}{\partial \boldsymbol{V}}   \quad   \frac{\partial \Eym}{\partial \boldsymbol{U}_{X}}   \right]
\end{equation}

with the functional form of the each covariate group in (\ref{eq:hme_marginal_effect_def})
defined as:

\begin{equation} \label{eq:ME_gating}
  \frac{\partial \Eym}{\partial \boldsymbol{U}_{Z}} = \frac{\partial \gateprod{g^{0}_{t}}{m}}{{\partial \boldsymbol{U}_{Z}}} \Eym
\end{equation}


\begin{equation} \label{eq:ME_expert}
  \frac{\partial \Eym}{\partial \boldsymbol{U}_{X}} = \gateprod{g^{0}_{t}}{m} \frac{\partial \Eym}{{\partial \boldsymbol{U}_{X}}}
\end{equation}


\begin{equation} \label{eq:ME_both}
  \frac{\partial \Eym}{\partial \boldsymbol{V}} = \frac{\partial \gateprod{g^{0}_{t}}{m}}{{\partial \boldsymbol{V}}} \Eym + \gateprod{g^{0}_{t}}{m} \frac{\partial \Eym}{{\partial \boldsymbol{V}}}
\end{equation}

Not matter how complex the model becomes, the researcher can always interpret
the estimated HME through a single vector of marginal effects of $\boldsymbol{T}$.
Of the four components in equations (\ref{eq:ME_gating}) - (\ref{eq:ME_both}),
three have already been established: $\Eym$ is the output from local
expert $m$, $\gateprod{g^{0}_{t}}{m_{t}}$ is the prior weight for
input $t$ for local expert $m$, and $\frac{\partial \Eym}{{\partial \boldsymbol{X}}}$
is the marginal effect of the local expert $m$ with respect to covariates $\boldsymbol{X}$.
What is left is the partial derivative of the gating network with respect to a
variable in that network $\frac{\partial \gateprod{g^{0}_{t}}{m}}{\partial \boldsymbol{z}_{t}}$.
Starting with equation (\ref{eq:gpath}), we take the partial with
respect to parameters in the gating matrix:

\begin{equation} \label{eq:hme_gate_marginal_effect_1}
  \boldsymbol{\delta}^{m}_{t} \equiv \frac{\partial \gateprod{g^{0}_{t}}{m}}{\partial \boldsymbol{z}_{t}} = \frac{\partial g^{0, i}_{t} g^{i|0, j}_{t} \cdots g^{k|\cdots|j|i|0, m}_{t}}{\partial \boldsymbol{z}_{t}}
\end{equation}

Applying the product rule yields:

\begin{equation} \label{eq:hme_gate_marginal_effect_2}
  \begin{split}
    \boldsymbol{\delta}^{m}_{t} &= \frac{\partial g^{0, i}_{t}}{\partial \boldsymbol{z}_{t}} g^{i|0, j}_{t} \cdots g^{k|\cdots|j|i|0, m}_{t}                       \\
                                     &+ g^{0, i}_{t} \frac{\partial g^{i|0, j}_{t}}{\partial \boldsymbol{z}_{t}} \cdots g^{k|\cdots|j|i|0, m}_{t} \\
                                     &+ \dots                                                                                     \\
                                     &+ g^{0, i}_{t} g^{i|0, j}_{t} \cdots \frac{\partial g^{k|\cdots|j|i|0, m}_{t}}{\partial \boldsymbol{z}_{t}} \\
  \end{split}
\end{equation}

Since

\begin{equation} \label{eq:node_marginal_effect}
  \frac{\partial g^{a, i}_{t}}{\partial \boldsymbol{z}_{t}} = g^{a, i}_{t} \left( \bw{a, i} - \sum_{j} g^{a, j}_{t} \bw{a, j} \right)^{\top} = g^{a, i}_{t} \left( \bw{a, i} - \mean{\boldsymbol{\omega}}^{a} \right)^{\top}
\end{equation}

we can substitute equation (\ref{eq:node_marginal_effect}) into
(\ref{eq:hme_gate_marginal_effect_2}) to arrive at:

\begin{equation} \label{eq:marginal_effects}
  \begin{split}
    \boldsymbol{\delta}^{m}_{t} &= \gateprod{g^{0}_{t}}{y^{m}} \left(\boldsymbol{\omega}^{0, i} + \boldsymbol{\omega}^{i|0, j} + \cdots + \boldsymbol{\omega}^{k|\cdots|j|i|0, m} - \left( \mean{\boldsymbol{\omega}}^{0} + \mean{\boldsymbol{\omega}}^{i|0} + \cdots + \mean{\boldsymbol{\omega}}^{k|\cdots|j|i|0} \right) \right)^{\top} \\
                                &= \gateprod{g^{0}_{t}}{y^{m}} ( \boldsymbol{W}^{m} )^{\top}
  \end{split}
\end{equation}

Looking closely at equation (\ref{eq:marginal_effects}), the instantaneous rate
of change of $\gateprod{g^{0}_{t}}{m}$ to small deviations of $\boldsymbol{z}_{t}$
has an interesting representation. The row vector $( \boldsymbol{W}^{m} )^{\top}$
mean differences the parameter values of each edge in the path from the
root node to expert $m$. This path is the \textit{only} path from the root
node to expert $m$. The average is weighted, as always, by the prior weight $\gateprod{g^{0}_{t}}{}$


\subsection{Delta method}

Using the delta method, we can approximate standard
errors for the marginal effects of the HME model. Starting with equation
(\ref{eq:hme_marginal_effect_def}) from the previous section, we break down
the gradient of the marginal effects with respect to the parameters by those 
in the gating network, $\boldsymbol{\Omega}$, and the parameters in the
expert regressions, $\boldsymbol{\beta}$. These results are collected in Table
 \ref{tbl:delta_method_gradients}.

 \bigskip

\begin{table}
  \begin{center}
    \begin{tabular}{| l | c c c |}
    \hline
                                                                                    & \underline{$\boldsymbol{U}_{Z}$}                                                            & \underline{$\boldsymbol{V}$}                                                                                                                                                                                           & \underline{$\boldsymbol{U}_{X}$}   \\ [2ex]
    $\frac{\partial \boldsymbol{\Delta_{t}^{m}}}{\partial \boldsymbol{\omega}^{a}}$ & $\frac{\partial \boldsymbol{\delta}^{m}_{t}}{\partial \boldsymbol{\omega}^{a}} \Eym$        & $\frac{\partial \boldsymbol{\delta}^{m}_{t}}{\partial \boldsymbol{\omega}^{a}} \Eym + \frac{\partial \gateprod{g^{0}_{t}}{m}}{\partial \boldsymbol{\omega}^{a}}  \frac{\partial \Eym}{{\partial \boldsymbol{V}}}$  & $\boldsymbol{0}$                   \\ [2ex]
    $\frac{\partial \boldsymbol{\Delta_{t}^{m}}}{\partial \boldsymbol{\beta}^{m}}$  & $\boldsymbol{0}$                                                                            & $\boldsymbol{\delta}^{m}_{t} \frac{\partial \Eym}{\partial \boldsymbol{\beta}^{m}} + \gateprod{g^{0}_{t}}{m}   \frac{\partial^{2} \Eym}{\partial \boldsymbol{V} \partial \boldsymbol{\beta}^{m}}$    & $\gateprod{g^{0}_{t}}{m}  \frac{\partial^{2} \Eym}{\partial \boldsymbol{U}_{X} \partial \boldsymbol{\beta}^{m}}$  \\ [1ex]
    \hline
    \end{tabular}
  \caption{\label{tbl:delta_method_gradients} Delta Method Gradient Cases}
  \end{center}
\end{table}


Again, many of the expressions in Table \ref{tbl:delta_method_gradients}
have already been defined in previous sections. The three expressions new to this
section are $\frac{\partial^{2} \Eym}{\partial \boldsymbol{X} \partial \boldsymbol{\beta}^{m}}$,
$\frac{\partial \boldsymbol{\delta}^{m}_{t}}{\partial \boldsymbol{\omega}^{a,i}}$,
and $\frac{\partial \gateprod{g^{0}_{t}}{f^{m}}}{\partial \bw{a,i}}$.
For the standard OLS regressions that are considered in this paper,
$\frac{\partial^{2} \Eym}{\partial \boldsymbol{X} \partial \boldsymbol{\beta}^{m}} = \boldsymbol{1}$.
Conceptually, $\frac{\partial \boldsymbol{\delta}^{m}_{t}}{\partial \boldsymbol{\omega}^{a,i}}$
describes how the marginal effects of the gating network change in response
to small changes in the parameters of $\boldsymbol{\Omega}$. The value of
$\frac{\partial \boldsymbol{\delta}^{m}_{t}}{\partial \boldsymbol{\omega}^{a,i}}$
depends on what role $\boldsymbol{\omega}^{a,i}$ plays in navigating an input
pattern from the root node to the expert $m$. In what follows, the indicator notation
introduced in Section \ref{sec:TheHessian} will be used where $\mathbb{1}\{a, i, m\}$
is equal to one if $\bw{a, i}$ is an explicit gating vector for expert $m$ and zero
if it is an implicit gating vector. In addition, $\odot$ will denote
element-wise multiplication for vectors/matrices and the sign function
is used where:

\begin{equation}
  \textrm{sgn}\{x\} = \begin{cases}
  1  & x > 0 \\
  0  & x = 0 \\
  -1 & x < 0
  \end{cases}
\end{equation}

With this notation in mind, the partial derivative of the marginal effects of an HME
with respect to a gate parameter vector is expressed as:

\begin{equation} \label{eq:delta_gate_partial_exp}
  \frac{\partial \boldsymbol{\delta}^{m}_{t}}{\partial \bw{a,i}} = \gateprod{g^{0}_{t}}{f^{m}} \left[ (\mathbb{1}\{a, i, m\} - g^{a,i}_{t})  +  \left[ \textrm{sgn} \left\{ \mathbb{1}\{a, i, m\} - 0.5 \right\}  (1 - g^{a,i}_{t}) \boldsymbol{W}^{m}  - \boldsymbol{G}^{a,i}  \right] \odot \boldsymbol{z}_{t}^{\top} \right]
\end{equation}

where $\boldsymbol{W}^{m}$ was first seen in equation (\ref{eq:marginal_effects}) and

\begin{equation}
  \boldsymbol{G}^{a,i} = \left\{ g^{a,i} (1 - g^{a,i}) \bw{a,i} - \sum_{j \neq i} g^{a,i} g^{a,j} \bw{a,j} \right\}
\end{equation}

The partial derivative of the prior weight with respect to gate
parameter vector $\bw{a, i}$ is:

\begin{equation}
  \frac{\partial \gateprod{g^{0}_{t}}{f^{m}}}{\partial \bw{a,i}} = \gateprod{g^{0}_{t}}{f^{m}} \left( \mathbb{1}\{a, i, m\} - g^{a,i} \right) \boldsymbol{z}_{t}^{\top}
\end{equation}

Standard errors for the marginal effects for the HME models can then be
constructed with the robust variance-covariance matrix from equation
(\ref{eq:robustgatevarcov}) and the collection of equations in this Section
that fully defines $\frac{\partial \boldsymbol{\Delta}}{\partial \boldsymbol{\theta}}$.


\begin{equation} \label{eq:std_errs_full_marginal_effects}
  Asy.Var \left[ \boldsymbol{ \hat{ \Delta } } \right] = \sum^{M}_{n=1}  \left( \frac{1}{T} \sum^{T}_{t=1} \frac{\partial \boldsymbol{\Delta}_{t}}{\partial \boldsymbol{\theta}_{n}} \right)     \boldsymbol{V}(\boldsymbol{\hat{\theta}})      \left( \frac{1}{T} \sum^{T}_{t=1} \frac{\partial \boldsymbol{\Delta}_{t}}{\partial \boldsymbol{\theta}_{n}} \right)^\top
\end{equation} 

\section{A simple example} \label{sec:SimpleExample}

In order to provide a concrete example of the concepts discussed previously,
the ME and HME models are demonstrated on a small and well known dataset
collected by \citet{Anderson1936} and popularized in the
statistics literature by \citet{Fisher1936}. Anderson collected
50 measurements each from three different species of iris flowers; the width and
length of both the petal and the sepal. Figure \ref{fig:Iris_dataset} provides a
basic view of the species specific clustering inherent in the data.
The work below uses the ME and HME architectures to estimate a flower's sepal
width using only its petal width as a predictor. The petal width will be used
as the sole covariate in the local linear expert regressions ($\boldsymbol{X}$) as well as
in the gating network ($\boldsymbol{Z}$). 

\begin{figure}[!ht]
  \includegraphics[width=\textwidth]{basic_iris_plot.jpeg}
  \caption{Three different iris species: Setosa
  (blue circles), Versicolor (orange triangles), Virginia (green crosses).
  Sepal width is on the vertical axis and petal width on the horizontal
  axis.}
  \label{fig:Iris_dataset}
\end{figure}



\begin{equation} \label{eq:HME_iris}
    sepal.width_{i} = \beta_{0} + \beta_{1} * petal.width_{i} + \varepsilon_{i} \enspace | \enspace \omega_{0} + \omega_{1} * petal.width_{i}
\end{equation}

The goal is to have the gating network of the 
models identify the inherent species-specific clustering without explicit
knowledge of each observation's species classification, and then fit an
appropriate local regression to the self-identified clusters. As a benchmark,
an OLS model is run where a flower's petal width is interacted with its species,
resulting in a species-specific estimation of sepal width.

\begin{equation} \label{eq:OLS_iris}
    sepal.width_{is} = \beta_{0,s} + \beta_{1, s} * petal.width_{is} + \varepsilon_{is}
\end{equation}

Two sets of regressions are run. Since the Versicolor and Virginica species
can be viewed as one larger cluster, a two-expert ME model is run
and compared to a benchmark OLS where Versicolor and Virginica are labelled
as the same species. A second set of regressions are run with three mixture
experts. When moving to the three expert model, there is now a choice
on what kind of gating architecture to employ. We can go deep by adding 
a gating network with depth two (HME), or we can go wide by keeping the
depth of the gating network at one (ME). Again, for comparative purposes, a benchmark
OLS regression is estimated for each species separately. Results are collected in Table \ref{tbl:Iris}.
Coefficients for local experts in the two expert ME regression match closely with the OLS
benchmark. The strong separation between the Setosa and Versicolor/Virginica
clusters makes it easy for the ME gating network to discriminate between the two using
just the Petal Width dimension. This task becomes a little more complicated when
considering all three species at the same time since there exists some overlap
between the Versicolor and Virginica clusters. When comparing the coefficients of the
local regressions (see Table \ref{tbl:Iris}), the HME architecture clearly
outperforms the ME architecture. While the ME model does obtain a larger
log-likelihood value than the OLS estimate, it fails to identify the three
separate species that are known to exist. The HME model, on the other hand,
naturally picks up on the three underlying clusters while also providing a superior
likelihood value. This speaks to one of the major caveats of using this class of model.
The likelihood value of an ME or HME can always been improved by adding more and
more experts, but this improvement should not be confused with the model gaining a
finer understanding of the underlying data generating process. It simply may start to
over-fit the data at hand.

\begin{landscape}
\begin{table} 
	\caption{Iris Dataset - OLS vs ME vs HME}
	\begin{threeparttable}
		\begin{tabular}[l]{l l l l l l l l l l l l}
  \cmidrule{1-12}

  & \multicolumn{4}{c}{2 Expert Mixture} & & \multicolumn{6}{c}{3 Expert Mixture} \\
  \cmidrule(r){2-5}  \cmidrule(r){7-12}
  & \multicolumn{2}{c}{OLS}  & \multicolumn{2}{c}{ME} & & \multicolumn{2}{c}{OLS}  & \multicolumn{2}{c}{HME}  & \multicolumn{2}{c}{ME} \\
  \cmidrule(r){2-3}  \cmidrule(r){4-5}  \cmidrule(r){7-8}  \cmidrule(r){9-10}  \cmidrule(r){11-12}
  & Coef. & SE & Coef. & SE && Coef. & SE & Coef. & SE & Coef. & SE     \\
  \cmidrule{1-12}
  
  Setosa \\
  \cmidrule(r){1-1}
  Const.             & 3.22 & 0.11** & 3.22 & 0.13** && 3.22  & 0.11** & 3.22 & 0.13** & 3.45 & 0.13**       \\
  Petal.Width        & 0.84 & 0.42*  & 0.95 & 0.49** && 0.84  & 0.41*  & 0.94 & 0.49   & 0.39 & 0.46         \\[0.3cm]
  
  Virginica \\
  \cmidrule(r){1-1}
  Const.             & --   & --     & --   & --     && 1.70  & 0.32** & 1.96 & 0.12** & 3.02 & 0.05**       \\
  Petal.Width        & --   & --     & --   & --     && 0.63  & 0.16** & 0.50 & 0.06** & 0.21 & 0.31         \\[0.3cm]
  
  Versicolor \\
  \cmidrule(r){1-1}
  Const.             & --   & --     & --   & --     && 1.37  & 0.29** & 1.15 & 0.12** & 2.13 & 0.09**       \\
  Petal.Width        & --   & --     & --   & --     && 1.05  & 0.22** & 1.29 & 0.09** & 0.44 & 0.06**       \\[0.3cm]

  Virg + Versi \\
  \cmidrule(r){1-1}
  Const.             & 2.13 & 0.13** & 2.13 & 0.09** && --    & --     & --   & --     & --   & --       \\
  Petal.Width        & 0.44 & 0.07** & 0.44 & 0.06** && --    & --     & --   & --     & --   & --       \\[0.3cm]

  AME \\
  \cmidrule(r){1-1}
  Petal.Width        & 0.57 & --     & 0.49 & --     && 0.84  & --     & 0.57 & --     & 0.62 & --         \\[0.3cm]


  Log-Like           & -35.5 & --    & -31.9 & --    && -29.3  & --    & -21.8& --     & -27.8 & -- \\
  N                  & 150   & --    & 150   & --    && 150    & --    & 150  & --     & 150   & -- \\

	\hline
		\end{tabular}
		\begin{tablenotes}
			\item{\footnotesize ** $p < 0.01$, * $p < 0.05$}
			\item{\footnotesize OLS regressions are modeled using equation (\ref{eq:OLS_iris})}
			\item{\footnotesize ME regressions are modeled using equation (\ref{eq:HME_iris}) and architecture $\boldsymbol{A}$ from figure \ref{fig:network_comparison}}
			\item{\footnotesize HME regressions are modeled using equation (\ref{eq:HME_iris}) and architecture $\boldsymbol{C}$ from figure \ref{fig:network_comparison}}
		\end{tablenotes} \label{tbl:Iris}
	\end{threeparttable}
\end{table}
\end{landscape}


\begin{figure}[!ht]
  \includegraphics[width=\textwidth]{Iris_fitted_regressions.jpeg}
  \caption{Comparison of the fitted experts between the ME and HME architectures
  applied to the Iris dataset. OLS regression estimates are drawn in solid lines.
  Although the HME and ME both achieve superior log-likelihood values compared to
  OLS, only the HME is able to identify the three iris species clusters.}
  \label{fig:Iris_fitted_regressions}
\end{figure}


\section{A Mincer Wage Equation} \label{sec:MincerWageEx}

For a more economically relevant example, we turn our attention to a common
topic in labor economics: the income return on an additional year of
education. At times called the "Mincer wage equation", our version of it
will be:

\begin{equation} \label{eq:mincer_equ}
  \log (wage) = \beta_{0} + \beta_{1} * \textnormal{Age} + \beta_{2} * \textnormal{Age}^{2} + \beta_{3} * \textnormal{YrsEdu} + \boldsymbol{\beta_{4}}\boldsymbol{X} + \varepsilon
\end{equation}

with $\boldsymbol{X}$ containing a set of individual-specific 
variables as well as a set of occupation-specific attributes. 
Our data will come from two sources. First, from the 2000 Census,
we devise a measure of the hourly (log) wage. In addition to income, we also collect information on age,
years of education (YrsEdu), job occupations codes, 
and a set of demographic identifiers indicating the race of the
individuals contained in the census sample. For the occupational codes, we use
the Standard Occupation Classification (SOC) codes from the
Occupation Information Network (ONet). Each occupation is
associated with a set of knowledge and skill-based
attributes describing what qualities are necessary to perform each
job suitably. A federally sponsored source, ONet details, "the knowledge,
skills, and abilities required as well as how the work is
performed in terms of tasks, work activities, and other descriptors"
(\cite{ONET}).

To link the occupational codes in the census data to the SOC codes used
by ONet, we use the cross walk provided by Sarah Porter (\cite{Crosswalk}).
This mapping is not one-to-one. When more than one SOC code points to a single census
code, we take the average of the SOC codes. After a quick but careful scan of the job attributes
available on ONet, the following four were selected. The footnotes
provide the full classification hierarchy listed on the website. 

\begin{enumerate}
  \item Social Perceptiveness \footnote{Skills - Social Skills - Social Perceptiveness}
  \item Design \footnote{Work Activities - Mental Processes - Analyzing Data or Information}
  \item Data Analytics \footnote{Knowledge - Design}
  \item Creative Thinking \footnote{Work Activities - Mental Processes - Thinking Creatively}
\end{enumerate}

The guiding principle for attribute selection was to choose a small but
diverse set of attributes that contrast well, with each attribute embodying
a human skill valued across industry, culture, and society. For these selected
attributes, ONet grades their relevance on a
100 point scale. Each attribute contains two scales, an "importance"
scale and a "level" scale. The importance scale denotes how critical the
attribute is to the occupation while the level indicates how much the
skill is required or needed to perform the occupation. To unify the two
measures, we follow the paper Prof Wijverberg gave me and take a
Cobb-Douglass style average with a 2/3's weight for importance and a 1/3
weight for the level scale.

\begin{figure}[t!]
  \includegraphics[width=\textwidth]{Job_characteristic_density.jpeg}
  \caption{Density estimates of ONet job characteristics broken down by
  sex. The job characteristics have been mean centered and scaled to
  have unit variance.}
  \label{fig:JobChar_vs_sex}
\end{figure}

The total number of individuals in the Census data numbers 105,796.
After applying our crosswalk, only 75,957 cases remain with complete
information across both datasets. Of those 75,957, roughly ten percent
(7,315) are randomly held-out and used as a test set to gauge
out-of-sample forecast performance across model specifications.
This leaves 68,642 individuals left as a training set. A statistical summary
of the covariates is provided in table \ref{tbl:census_cov_summary}.

\begin{table} \centering
  \caption{Summary Statistics}
  \begin{threeparttable}
    \begin{tabular}[l]{l r r r r}
  \cmidrule{1-5}

              & 25\%   & Mean & 50\% & 75\%    \\
  \cmidrule{1-5}
  Wage (hr)   & 9.20  & 15.82 & 13.32  & 19.44 \\
  Yrs Edu     & 12.00 & 13.78 & 14.00  & 16.00 \\
  Age         & 30.00 & 39.15 & 39.00  & 48.00 \\
  Age16       & 14.00 & 23.15 & 23.00  & 32.00 \\
  Female      & --    & 40.47 & --     & --    \\
  Af Amer     & --    &  8.62 & --     & --    \\
  Indian      & --    &  1.05 & --     & --    \\
  White       & --    & 77.00 & --     & --    \\
  Hispanic    & --    & 10.00 & --     & --    \\
  Asian       & --    &  3.36 & --     & --    \\
  Creative    & 46.65 & 53.30 & 53.82  & 58.94 \\
  Design      & 15.00 & 30.58 & 26.33  & 38.97 \\
  Analytic    & 44.01 & 52.63 & 52.68  & 62.18 \\
  Perceptive  & 41.15 & 50.49 & 46.03  & 59.84 \\

  \hline
    \end{tabular}
    \begin{tablenotes}
      \item{\footnotesize N = 68,642}
    \end{tablenotes} \label{tbl:census_cov_summary}
  \end{threeparttable}
\end{table}

A natural question to consider as a researcher is where to put the variable(s)
of interest while performing an HME estimation. \cite{JiangTanner2000}
provide their proof of model consistency for HME of GLM's for the case where
all covariates appear in the gating network as well as the experts. We will
call this the \textit{full} specifications:

\begin{equation} \label{eq:full_formula}
  log(wage) = Age + YrsEdu + Sex +  Race + Occ \; | \; Age + YrsEdu + Sex +  Race + Occ
\end{equation}

We will compare this \textit{full} specification to two others. A
\textit{mid} specification where the local experts contain age and
years of education while removing demographic indicators:

\begin{equation} \label{eq:mid_formula}
  log(wage) = Age + YrsEdu \; | \; Age + YrsEdu + Sex +  Race + Occ
\end{equation}

And finally a \textit{minimal} specification where our core variable
of interest, years of education, appears solely in the gating network.

\begin{equation} \label{eq:min_formula}
  log(wage) = Age \; | \; Age + YrsEdu + Sex +  Race + Occ
\end{equation}

For comparative purposes, we estimate several different regressions across three
different dimensions: model architectures (ME vs HME), the number of experts,
and the regression specification (equations (\ref{eq:full_formula}) -
(\ref{eq:min_formula})). Table \ref{tbl:model_comparison} presents a view of these
results across those dimensions. After looking at the results,
two themes emerge. First, there is a clear advantage to using the HME structure
if the aim is to maximize the likelihood value. The HME structure shows consistent
improvement across specifications as the number of experts increase, while the ME
struggles to improve the likelihood value if there is only one gating split.
This increase in efficiency is most likely due
to the HME's more refined gating architecture, whose recursive partitioning is more
effective at finding the next improvement in the parameter vector than the single
multinomial split in the ME. As for the second theme, it's best to give the expert
regressions as much information as possible. The Full specification clearly outperforms
the Mid specification, which outperforms the Min specification. Referencing table
\ref{tbl:model_comparison}, if one holds the architecture and the number of experts
constant, the performance metrics show clear improvement as the regression specification
adds more explanatory variables.

Turning attention to the main variable of focus, table \ref{tbl:YrsEdu_coef} provides a 
comparison of the average marginal effect for \textit{YrsEdu} across the same dimensions
explored for the performance metrics. There is a noticeable change across model specifications.
Compared to the OLS coefficient of 0.76, the Min specification,
which includes \textit{YrsEdu} only in the gating network, underestimates the returns
to education. The Mid specification, which includes \textit{Age} and
\textit{YrsEdu} in the expert regressions as well as the gating network,
overestimates the returns to education in all the models except the HME with four
and five experts. The Full specification, which has the entire suite of
variables in both places, matches most closely to the OLS estimate across the estimated
models.

For our census sample, estimating up to five experts is pretty extreme.
It's rather unlikely that there exists more than one distinct cluster, let alone two\footnote{Testing if a (H)ME model is even necessary would be a valuable addition to this paper}.
Because of this, a deeper analysis of the regression results are only explored
for the three models that have the least complexity/experts. We first estimate equation (\ref{eq:mincer_equ}) 
for a two expert model. At this specification, there is no distinction
between the HME and ME. A three expert model is then estimated for these two
respective architectures to asses if different conclusions to the estimated
Mincer equations arise. Results for these regressions are collected in tables \ref{tbl:2E_full_regressions_results},
\ref{tbl:3W_full_regressions_results}, and \ref{tbl:3D_full_regressions_results}
and complimented by graphs \ref{tbl:ME2_sample_comparison}, \ref{tbl:ME3_sample_comparison},
and \ref{tbl:HME3_sample_comparison}, which provide mean and median values for the subset of individuals
in the census sample that are attributed to each expert based on the value of
their posterior weights\footnote{For example, observation $i$ is assigned to expert $j$ if the posterior vector's largest value is the $j$-th index: $\argmax \boldsymbol{h}_{i} = h_{ij}$}.

Broadly speaking, all three models explored share the same macro view of the data.
On the right side of tables \ref{tbl:2E_full_regressions_results},
\ref{tbl:3W_full_regressions_results}, and \ref{tbl:3D_full_regressions_results}
are a group of columns titled '(H)ME Marginal Effects'.
Here the marginal effects of the model can be broken down and attributed to the
gating network or the expert regressions. "Both", "Experts", and "Gates" refers to marginal effects referenced by equations
(\ref{eq:ME_both}), (\ref{eq:ME_expert}), and (\ref{eq:ME_gating}),
respectively. The values are fairly consistent across variables and model
architectures with the coefficients for \textit{Age} and it's square a modest exception,
ranging from 0.028 (HME) to 0.042 (2-Expert ME) for \textit{Age}. Notice also that the
marginal effects from the expert regressions are the lion's share of total marginal effect,
ranging from one to two orders of magnitude larger than marginal effects for the gating
network.

When left to segment the data set on it's own, the two expert ME model estimates two different
wage equations, one for the majority of the population that tends to be
older, whiter, and more educated (see table \ref{tbl:ME2_sample_comparison}), and a second
smaller popular that is more diverse, significantly young, with less education on average.
The difference between the average age of the two populations is noticeable and
might play a role behind the marginal effects for \textit{Age} moving around as
much as it does. This younger cohort breaking off from the bulk of the sample
repeats for both 3-expert models as well. Interestingly, the 3-expert models then share a
further partition of the sample around age, with a third older cohort separating itself
from the smaller sample. When taken together with the regression diagnostics, the model
suggests that returns to education evolve over an individual's lifetime. When young,
the returns to education are at their smallest and then expand during main earning years
of middle age. Returns then dip slightly as individuals come closer to retirement age.

When looking at the occupational attributes there is similar agreement between
the estimated models. The marginal effects for all three are in close proximity between
the ME and HME models. Those individuals who specialize in performing analytics
enjoy the greatest hourly rate (0.126 - 0.128). Design (0.074 to 0.081) and 
Perceptive (0.053 to 0.057) attributes get a smaller bump to the their hourly
wage while Creative types (-0.044 to -0.043) clearly have alternative motivation than
monetary gain.


\section{Conclusion} \label{sec:Conclusion}

In this article, a novel mixture model is explored that borrows equally from the
economic and deep learning fields. A flexible (and optionally deep) gating network
is used to learn the latent structure of a dataset and then apply local
regressions to that latent structure. Robust standard errors and closed form expressions
for marginal effects were developed and demonstrated on two different datasets. 


\begin{table} \centering
  \caption{Comparing Complexity, Architecture, and Regression Specification}
  \begin{threeparttable}
    {\footnotesize
    \begin{tabular}[r]{l l l r r r r}
  \cmidrule{1-7}
         &       &         &  \multicolumn{4}{c}{Performance Metrics} \\ 
   \cmidrule(l){4-7}
Specification & Architecture  & Experts & Log-Lik & AIC    & BIC    & MSE   \\ 
  \cmidrule{1-7}

Full     &  ME   &    2    & -0.541  & 1.082  & 1.088  & 0.182 \\
         &  ME   &    3    & -0.526  & 1.053  & 1.062  & 0.182 \\
         &  ME   &    4    & -0.537  & 1.078  & 1.091  & 0.181 \\
         &  ME   &    5    & -0.535  & 1.073  & 1.089  & 0.182 \\
         &  HME  &    3    & -0.525  & 1.052  & 1.061  & 0.182 \\
         &  HME  &    4    & -0.515  & 1.034  & 1.047  & 0.181 \\
         &  HME  &    5    & \iu{-0.505}  & \iu{1.015}  & \iu{1.031}  & \iu{0.178} \\
         &       &         &         &        &        &       \\
Mid      &  ME   &    2    & -0.560  & 1.120  & 1.123  & 0.185 \\
         &  ME   &    3    & -0.558  & 1.117  & 1.123  & 0.186 \\
         &  ME   &    4    & -0.581  & 1.163  & 1.171  & 0.192 \\
         &  ME   &    5    & -0.590  & 1.182  & 1.192  & 0.199 \\
         &  HME  &    3    & -0.541  & 1.083  & 1.088  & 0.184 \\
         &  HME  &    4    & -0.528  & 1.057  & 1.065  & 0.183 \\
         &  HME  &    5    & \it{-0.519}  & \it{1.039} & \it{1.050}  & \it{0.182} \\
         &       &         &         &        &        &       \\
Min      &  ME   &    2    & -0.596  & 1.192  & 1.195  & 0.192 \\
         &  ME   &    3    & -0.587  & 1.176  & 1.181  & 0.192 \\
         &  ME   &    4    & -0.629  & 1.260  & 1.268  & 0.211 \\
         &  ME   &    5    & -0.564  & 1.131  & 1.140  & 0.189 \\
         &  HME  &    3    & -0.581  & 1.163  & 1.168  & 0.190 \\
         &  HME  &    4    & -0.546  & 1.094  & 1.101  & 0.182 \\
         &  HME  &    5    & \it{-0.524}  & \it{1.049}  & \it{1.059}  & \it{0.182} \\
        \cmidrule(l){1-7}
    \end{tabular}
    }
    \begin{tablenotes}
      \item{\footnotesize \textbf{Note:} Log-Likelihood, AIC, and BIC are divided by the sample size: 68,642.
      Italicized entries are the winning values within specification while underlined entries are the best values across all three specifications.
      }
      \item{\footnotesize \textbf{Note:} The MSE is calculated from a hold-out test set with sample size: 7,315}
      \item{\footnotesize \textbf{Note:} After looking at the results, two themes emerge. \textbf{One}, there is a clear advantage to using the HME structure if the aim is to maximize the likelihood value.
      The HME structure shows consistent improvement across specifications as the number of experts increases,
      while the ME struggles to match this consistency. \textbf{Two}, give the expert regressions as much information as possible. The Full specification clearly outperforms the Mid and Min specifications across the board.
      }
      \item{\footnotesize }
    \end{tablenotes} \label{tbl:model_comparison}
  \end{threeparttable}
\end{table}



\begin{table} \centering
  \caption{Returns to Years of Education}
  \begin{threeparttable}
    \begin{tabular}[l]{r r r r r}
  \cmidrule{1-5}
        &         & \multicolumn{3}{c}{Avg. Marginal Effect} \\ 
  \cmidrule(r){3-5}
  Depth & Experts & Min   & Mid   & Full      \\
  \cmidrule{1-5}

  ME      & 2       & 0.051 & 0.082 & 0.076     \\
  ME      & 3       & 0.051 & 0.081 & 0.074     \\
  ME      & 4       & 0.039 & 0.085 & 0.075     \\
  ME      & 5       & 0.063 & 0.095 & 0.076     \\
  HME     & 3       & 0.063 & 0.080 & 0.073     \\
  HME     & 4       & 0.063 & 0.078 & 0.073     \\
  HME     & 5       & 0.068 & 0.075 & 0.069     \\

  \hline
    \end{tabular}
    \begin{tablenotes}
      \item{\footnotesize \textbf{Note:} OLS coef: 0.076}
      \item{\footnotesize \textbf{Note:} There is a noticeable change across in the marginal return to an extra year of education.
      Compared to the OLS coefficient of 0.76, the Min specification, which includes \textit{YrsEdu} only in the gating network, underestimates the returns to education.
      The Mid specification, which includes \textit{Age} and \textit{YrsEdu} in the expert regressions as well as the gating network, overestimates the returns to education in all the models except the HME with four and five experts.
      The Full specification, which has the entire suite of variables in both places, matches most closely to the OLS estimate across the estimated models.}
    \end{tablenotes} \label{tbl:YrsEdu_coef}
  \end{threeparttable}
\end{table}



\begin{landscape}
\begin{table} \centering
  \caption{Regression Results: Two-Expert, Full Parameter Specification}
    \begin{threeparttable}
      \begin{tabular}[l]{l r l r l c r l c r l r l r l}

\cmidrule{1-15}
& \multicolumn{4}{c}{ME Regressions$^{1}$} &&  \multicolumn{2}{c}{OLS$^{2}$} && \multicolumn{6}{c}{ME Marginal Effects$^{3}$} \\
\cmidrule(l){2-5}    \cmidrule(l){7-8}     \cmidrule(l){10-15}
& \multicolumn{2}{c}{Coef.} &  \multicolumn{2}{c}{Coef.}  && \multicolumn{2}{c}{Coef.} && \multicolumn{2}{c}{Both}    &  \multicolumn{2}{c}{Experts}  &  \multicolumn{2}{c}{Gates} \\
                \cmidrule(l){2-3} \cmidrule(l){4-5} \cmidrule(l){7-8} \cmidrule(l){10-11} \cmidrule(l){12-13} \cmidrule(l){14-15}
Intercept     &  1.231 & *      &  1.494 & *     &&  1.241 & *     &&  1.225 & *        &  1.260 & *        & -0.040 &        \\
Age16           &  0.032 & *      &  0.068 & *     &&  0.035 & *     &&  0.042 &          &  0.038 & *        &  0.004 &        \\
Age16sq         & -0.000 & *      & -0.002 & *     && -0.001 & *     && -0.001 &          & -0.001 & *        & -0.000 &        \\
YrsEduc         &  0.082 & *      &  0.036 & *     &&  0.076 & *     &&  0.076 & *        &  0.075 & *        &  0.000 &        \\
Female          & -0.244 & *      & -0.032 & *     && -0.215 & *     && -0.209 & *        & -0.207 & *        & -0.002 &        \\
Af Amer         & -0.076 & *      & -0.045 & *     && -0.076 & *     && -0.076 & *        & -0.071 & *        & -0.005 &        \\
Indian          & -0.081 & *      &  1.390 & *     && -0.091 & *     && -0.085 & +        & -0.079 & *        & -0.005 &        \\
Asian           & -0.045 & *      &  0.036 & *     && -0.032 & *     && -0.024 &          & -0.028 & *        &  0.003 &        \\
Hisp            & -0.121 & *      & -0.082 & *     && -0.106 & *     && -0.112 & *        & -0.112 & *        & -0.000 &        \\
Creativity      & -0.054 & *      & -0.008 & *     && -0.046 & *     && -0.044 & *        & -0.045 & *        &  0.002 &        \\
Design          &  0.080 & *      &  0.078 & *     &&  0.082 & *     &&  0.081 & *        &  0.080 & *        &  0.001 &        \\
Analytics       &  0.133 & *      &  0.112 & *     &&  0.131 & *     &&  0.126 & *        &  0.129 & *        & -0.003 &        \\
Perceptive      &  0.063 & *      & -0.013 & *     &&  0.058 & *     &&  0.053 & *        &  0.049 & *        &  0.004 &        \\
Log-Variance    & -1.651 & *      & -2.682 & *     &&  --    &       &&  --    &          &  --    &          &  --    &        \\
                \cmidrule(l){2-3} \cmidrule(l){4-5} \cmidrule(l){7-8} \cmidrule(l){10-11} \cmidrule(l){12-13} \cmidrule(l){14-15}
Share$^{4}$:          &  0.826 &        &  0.174 &       &&  1.000 &       &&  --    &          &  --    &          &  --    &        \\
\cmidrule{1-15}
      \end{tabular}

      \begin{tablenotes}
        \item Signif. Codes: 0.01 '*', 0.05 '+', 0.1 '-'
        \item Log-Likelihood: ME -0.541, OLS -0.558
        \item $^{1}$ Fitted coefficients from the two-expert model with the full parameter specification from equation (\ref{eq:full_formula})
        \item $^{2}$ Fitted coefficients from an OLS regression. These coefficient values can be compared to the HME coefficients to their left as well as to the marginal values to their right
        \item $^{3}$ Marginal effects for the HME model. Both, Experts, and Gates refers to marginal effects referenced by equations (\ref{eq:std_errs_full_marginal_effects}), (\ref{eq:ME_expert_only}), and (\ref{eq:ME_gating_only}), respectively.
        \item $^{4}$ The share is calculated by summing the posterior weights across observations for each expert.

      \end{tablenotes} \label{tbl:2E_full_regressions_results}


    \end{threeparttable}

\end{table}
\end{landscape}


\begin{table} \centering
  \caption{Sample Mean Comparison: Two-Expert ME}
  \begin{threeparttable}
    \begin{tabular}[l]{l r r r r}
  \cmidrule{1-5}
  Share:$^{1}$& \multicolumn{2}{c}{(0.826)} & \multicolumn{2}{c}{(0.174)} \\
              & Mean & Median & Mean & Median \\
  \cmidrule{1-5}
  Wage (hr)   &   2.679 &   2.681 &  2.175 &  2.197 \\
  Age         &  25.814 &  25.000 &  6.965 &  7.000 \\
  Age16       & 759.812 & 625.000 & 62.478 & 49.000 \\
  Female      &   0.408 &   0.000 &  0.386 &  0.000 \\
  Af Amer     &   0.084 &   0.000 &  0.101 &  0.000 \\
  Indian      &   0.009 &   0.000 &  0.018 &  0.000 \\
  White       &   0.778 &   1.000 &  0.698 &  1.000 \\
  Hispanic    &   0.037 &   0.000 &  0.028 &  0.000 \\
  Asian       &   0.091 &   0.000 &  0.155 &  0.000 \\
  YrsEduc     &  13.916 &  14.000 & 12.974 & 12.000 \\
  Creative    &  -0.191 &  -0.137 & -0.464 & -0.542 \\
  Design      &  -0.344 &  -0.535 & -0.442 & -0.635 \\
  Analytic    &  -0.196 &  -0.247 & -0.499 & -0.550 \\
  Perceptive  &   0.230 &   0.127 & -0.233 & -0.532 \\
  \cmidrule{1-5}
  N           &      -- &  58,939 &     -- &  9,703 \\
  \hline
    \end{tabular}
    \begin{tablenotes}
      \item{\footnotesize $^{1}$ The share is calculated by summing the 
      posterior weights across observations for each expert.}
      \item{\footnotesize \textbf{Note:} Mean and median values are applied to individuals
      in the census sample that are classified based on the value of their posterior weights.
      For example, observation $i$ is assigned to expert $j$ if the posterior vector's
      largest value is the $j$-th index: $\argmax \boldsymbol{h}_{i} = h_{ij}$}
    \end{tablenotes} \label{tbl:ME2_sample_comparison}
  \end{threeparttable}
\end{table}


\begin{landscape}
\begin{table} \centering
  \caption{Regression Results: Wide Three-Expert, Full Parameter Specification}
    \begin{threeparttable}
      \begin{tabular}[l]{l r l r l r l c r l c r l r l r l}

\cmidrule{1-17}
& \multicolumn{6}{c}{ME Regressions$^{1}$} &&  \multicolumn{2}{c}{OLS$^{2}$} && \multicolumn{6}{c}{ME Marginal Effects$^{3}$} \\
\cmidrule(l){2-7}    \cmidrule(l){9-10}     \cmidrule(l){12-17}
& \multicolumn{2}{c}{Coef.} &  \multicolumn{2}{c}{Coef.} &  \multicolumn{2}{c}{Coef.}  && \multicolumn{2}{c}{Coef.} && \multicolumn{2}{c}{Both}    &  \multicolumn{2}{c}{Experts}  &  \multicolumn{2}{c}{Gates} \\
                \cmidrule(l){2-3} \cmidrule(l){4-5} \cmidrule(l){6-7} \cmidrule(l){9-10} \cmidrule(l){12-13} \cmidrule(l){14-15} \cmidrule(l){16-17}
Intercept     &  1.379 & *      &  1.574 & *      &  0.562 & *      &&  1.241 & *     &&  1.367 &          &  1.340 & *        &  0.032 &        \\
Age16           &  0.021 & *      &  0.045 & *      &  0.060 & *      &&  0.035 & *     &&  0.029 &          &  0.027 & *        &  0.002 &        \\
Age16sq         & -0.000 & *      & -0.001 & *      & -0.001 & *      && -0.001 & *     && -0.000 &          & -0.000 & *        &  0.000 &        \\
YrsEduc         &  0.082 & *      &  0.032 & *      &  0.080 & *      &&  0.076 & *     &&  0.074 &          &  0.077 & *        & -0.002 &        \\
Female          & -0.251 & *      & -0.022 & *      & -0.149 & *      && -0.215 & *     && -0.206 &          & -0.218 & *        &  0.012 &        \\
Af Amer         & -0.084 & *      & -0.056 & *      & -0.054 & -      && -0.076 & *     && -0.076 &          & -0.078 & *        &  0.002 &        \\
Indian          & -0.105 & *      & -0.046 & *      &  0.010 &        && -0.091 & *     && -0.091 &          & -0.090 & *        & -0.002 &        \\
Asian           & -0.030 & *      &  0.057 & *      & -0.091 & *      && -0.032 & *     && -0.024 &          & -0.025 & *        &  0.001 &        \\
Hisp            & -0.136 & *      & -0.061 & *      &  0.071 & +      && -0.106 & *     && -0.107 &          & -0.111 & *        &  0.004 &        \\
Creativity      & -0.038 & *      & -0.022 & *      & -0.177 & *      && -0.046 & *     && -0.044 &          & -0.047 & *        &  0.003 &        \\
Design          &  0.080 & *      &  0.080 & *      & -0.037 & *      &&  0.082 & *     &&  0.075 &          &  0.071 & *        &  0.004 &        \\
Analytics       &  0.123 & *      &  0.110 & *      &  0.196 & *      &&  0.131 & *     &&  0.128 &          &  0.128 & *        &  0.000 &        \\
Perceptive      &  0.060 & *      & -0.008 & *      &  0.168 & *      &&  0.058 & *     &&  0.057 &          &  0.061 & *        & -0.004 &        \\
Log-Variance    & -1.893 & *      & -2.891 & *      & -0.627 & *      &&  --    &       &&  &          &  &          &  &        \\
\cmidrule(l){2-3} \cmidrule(l){4-5} \cmidrule(l){6-7} \cmidrule(l){9-10} \cmidrule(l){12-13} \cmidrule(l){14-15} \cmidrule(l){16-17}
Share$^{4}$:          & 0.809  &        & 0.111  &      & 0.080 &           &&  1.000 &       &&  --    &          &  --    &          &  --    &        \\
\cmidrule{1-17}
      \end{tabular}

      \begin{tablenotes}
        \item Signif. Codes: 0.01 '*', 0.05 '+', 0.1 '-'
        \item Log-Likelihood: ME -0.526, OLS -0.558
        \item $^{1}$ Fitted coefficients from the three-expert model with the full parameter specification from equation (\ref{eq:full_formula})
        \item $^{2}$ Fitted coefficients from an OLS regression. These coefficient values can be compared to the HME coefficients to their left as well as to the marginal values to their right
        \item $^{3}$ Marginal effects for the HME model. Both, Experts, and Gates refers to marginal effects referenced by equations (\ref{eq:std_errs_full_marginal_effects}), (\ref{eq:ME_expert_only}), and (\ref{eq:ME_gating_only}), respectively.
        \item $^{4}$ The share is calculated by summing the posterior weights across observations for each expert.

      \end{tablenotes} \label{tbl:3W_full_regressions_results}


    \end{threeparttable}

\end{table}
\end{landscape}


\begin{table} \centering
  \caption{Sample Mean Comparison: Wide Three-Expert HME}
  \begin{threeparttable}
    \begin{tabular}[l]{l r r r r r r}
  \cmidrule{1-7}
  Share:$^{1}$& \multicolumn{2}{c}{(0.809)} & \multicolumn{2}{c}{(0.111)} & \multicolumn{2}{c}{(0.080)} \\
              & Mean & Median & Mean & Median & Mean & Median \\
  \cmidrule{1-7}
  Wage (hr)   &   2.664 &   2.667 &  2.106 &  2.096 &   2.549 &   2.221 \\
  Age         &  24.916 &  24.000 &  5.830 &  6.000 &  27.827 &  28.000 \\
  Age16       & 722.355 & 576.000 & 41.789 & 36.000 & 904.103 & 784.000 \\
  Female      &   0.420 &   0.000 &  0.301 &  0.000 &   0.250 &   0.000 \\
  Af Amer     &   0.090 &   0.000 &  0.060 &  0.000 &   0.064 &   0.000 \\
  Indian      &   0.010 &   0.000 &  0.012 &  0.000 &   0.010 &   0.000 \\
  Hispanic    &   0.036 &   0.000 &  0.020 &  0.000 &   0.102 &   0.000 \\
  Asian       &   0.100 &   0.000 &  0.114 &  0.000 &   0.045 &   0.000 \\
  YrsEduc     &  13.802 &  14.000 & 13.101 & 12.000 &  15.837 &  16.000 \\
  Creative    &  -0.209 &  -0.141 & -0.422 & -0.456 &  -0.195 &  -0.282 \\
  Design      &  -0.344 &  -0.535 & -0.387 & -0.535 &  -0.765 &  -0.860 \\
  Analytic    &  -0.218 &  -0.264 & -0.472 & -0.412 &  -0.072 &   0.049 \\
  Perceptive  &   0.177 &   0.127 & -0.122 & -0.455 &   0.851 &   0.877 \\
  \cmidrule{1-7}
  N           &      -- &  60,396 &     -- &  6,603 &      -- &   1,643 \\
  \hline
    \end{tabular}
    \begin{tablenotes}
      \item{\footnotesize $^{1}$ The share is calculated by summing the 
      posterior weights across observations for each expert.}
      \item{\footnotesize \textbf{Note:} Mean and median values are applied to individuals
      in the census sample that are classified based on the value of their posterior weights.
      For example, observation $i$ is assigned to expert $j$ if the posterior vector's
      largest value is the $j$-th index: $\argmax \boldsymbol{h}_{i} = h_{ij}$}
    \end{tablenotes} \label{tbl:ME3_sample_comparison}
  \end{threeparttable}
\end{table}


\begin{landscape}
  \begin{table} \centering
    \caption{Regression Results: Deep Three-Expert, Full Parameter Specification}
      \begin{threeparttable}
        \begin{tabular}[l]{l r l r l r l c r l c r l r l r l}
  
  \cmidrule{1-17}
  & \multicolumn{6}{c}{HME Regressions$^{1}$} &&  \multicolumn{2}{c}{OLS$^{2}$} && \multicolumn{6}{c}{HME Marginal Effects$^{3}$} \\
  \cmidrule(l){2-7}    \cmidrule(l){9-10}     \cmidrule(l){12-17}
  & \multicolumn{2}{c}{Coef.} &  \multicolumn{2}{c}{Coef.} &  \multicolumn{2}{c}{Coef.}  && \multicolumn{2}{c}{Coef.} && \multicolumn{2}{c}{Both}    &  \multicolumn{2}{c}{Experts}  &  \multicolumn{2}{c}{Gates} \\
                  \cmidrule(l){2-3} \cmidrule(l){4-5} \cmidrule(l){6-7} \cmidrule(l){9-10} \cmidrule(l){12-13} \cmidrule(l){14-15} \cmidrule(l){16-17}
  Intercept       &  1.404 & *      &  1.559 & *      &  0.898 & *      &&  1.241 & *     &&  1.393 &          &  1.382 & *        &  0.011 &        \\
  Age16           &  0.020 & *      &  0.050 & *      &  0.044 & *      &&  0.035 & *     &&  0.028 &          &  0.026 & *        &  0.003 &        \\
  Age16sq         & -0.000 & *      & -0.001 & *      & -0.001 & *      && -0.001 & *     && -0.000 &          & -0.000 & *        &  0.000 &        \\
  YrsEduc         &  0.082 & *      &  0.034 & *      &  0.074 & *      &&  0.076 & *     &&  0.073 &          &  0.075 & *        & -0.001 &        \\
  Female          & -0.257 & *      & -0.034 & *      & -0.131 & *      && -0.215 & *     && -0.209 &          & -0.217 & *        &  0.008 &        \\
  Af Amer         & -0.086 & *      & -0.048 & *      & -0.041 &        && -0.076 & *     && -0.076 &          & -0.077 & *        &  0.001 &        \\
  Indian          & -0.113 & *      & -0.057 & *      &  0.043 &        && -0.091 & *     && -0.100 &          & -0.093 & *        & -0.007 &        \\
  Asian           & -0.033 & *      &  0.058 & *      & -0.062 & +      && -0.032 & *     && -0.025 &          & -0.023 & *        & -0.001 &        \\
  Hisp            & -0.143 & *      & -0.066 & *      &  0.077 & *      && -0.106 & *     && -0.111 &          & -0.114 & *        &  0.003 &        \\
  Creativity      & -0.042 & *      & -0.021 & *      & -0.136 & *      && -0.046 & *     && -0.043 &          & -0.047 & *        &  0.004 &        \\
  Design          &  0.080 & *      &  0.068 & *      & -0.048 & *      &&  0.082 & *     &&  0.074 &          &  0.068 & *        &  0.006 &        \\
  Analytics       &  0.124 & *      &  0.112 & *      &  0.183 & *      &&  0.131 & *     &&  0.128 &          &  0.127 & *        &  0.000 &        \\
  Perceptive      &  0.063 & *      & -0.003 &        &  0.135 & *      &&  0.058 & *     &&  0.056 &          &  0.061 & *        & -0.004 &        \\
  Log-Variance    & -1.895 & *      & -2.791 & *      & -0.622 & *      &&  --    &       &&  &          &  &          &  &        \\
  \cmidrule(l){2-3} \cmidrule(l){4-5} \cmidrule(l){6-7} \cmidrule(l){9-10} \cmidrule(l){12-13} \cmidrule(l){14-15} \cmidrule(l){16-17}
  Share$^{4}$:          & 0.783  &        & 0.133  &      & 0.084 &           &&  1.000 &       &&  --    &          &  --    &          &  --    &        \\
  \cmidrule{1-17}
        \end{tabular}
  
        \begin{tablenotes}
          \item Signif. Codes: 0.01 '*', 0.05 '+', 0.1 '-'
          \item Log-Likelihood: HME -0.525, OLS -0.558
          \item $^{1}$ Fitted coefficients from the three-expert model with the full parameter specification from equation (\ref{eq:full_formula})
          \item $^{2}$ Fitted coefficients from an OLS regression. These coefficient values can be compared to the HME coefficients to their left as well as to the marginal values to their right
          \item $^{3}$ Marginal effects for the HME model. Both, Experts, and Gates refers to marginal effects referenced by equations (\ref{eq:std_errs_full_marginal_effects}), (\ref{eq:ME_expert_only}), and (\ref{eq:ME_gating_only}), respectively.
          \item $^{4}$ The share is calculated by summing the posterior weights across observations for each expert.
  
        \end{tablenotes} \label{tbl:3D_full_regressions_results}
  
  
      \end{threeparttable}
  
  \end{table}
  \end{landscape}


  \begin{table} \centering
    \caption{Sample Mean Comparison: Deep Three-Expert HME}
    \begin{threeparttable}
      \begin{tabular}[l]{l r r r r r r}
    \cmidrule{1-7}
    Share:$^{1}$& \multicolumn{2}{c}{(0.783)} & \multicolumn{2}{c}{(0.133)} & \multicolumn{2}{c}{(0.084)} \\
                & Mean & Median & Mean & Median & Mean & Median \\
    \cmidrule{1-7}
    Wage (hr)   &   2.683 &   2.676 &  2.137 &  2.140 &   2.432 &   2.075 \\
    Age         &  25.523 &  25.000 &  6.494 &  7.000 &  26.913 &  27.000 \\
    Age16       & 746.250 & 625.000 & 50.858 & 49.000 & 873.924 & 729.000 \\
    Female      &   0.414 &   0.000 &  0.358 &  0.000 &   0.313 &   0.000 \\
    Af Amer     &   0.088 &   0.000 &  0.073 &  0.000 &   0.075 &   0.000 \\
    Indian      &   0.010 &   0.000 &  0.016 &  0.000 &   0.018 &   0.000 \\
    White       &   0.770 &   1.000 &  0.753 &  1.000 &   0.749 &   1.000 \\
    Hispanic    &   0.036 &   0.000 &  0.027 &  0.000 &   0.101 &   0.000 \\
    Asian       &   0.096 &   0.000 &  0.131 &  0.000 &   0.057 &   0.000 \\
    YrsEduc     &  13.846 &  14.000 & 13.077 & 12.000 &  15.378 &  16.000 \\
    Creative    &  -0.198 &  -0.137 & -0.444 & -0.508 &  -0.201 &  -0.282 \\
    Design      &  -0.330 &  -0.530 & -0.477 & -0.635 &  -0.757 &  -0.859 \\
    Analytic    &  -0.206 &  -0.253 & -0.471 & -0.412 &  -0.161 &  -0.007 \\
    Perceptive  &   0.185 &   0.127 & -0.082 & -0.308 &   0.756 &   0.877 \\
    \cmidrule{1-7}
    N           &      -- &  58,429 &     -- &  8,674 &      -- &   1,539 \\
    \hline
      \end{tabular}
      \begin{tablenotes}
        \item{\footnotesize $^{1}$ The share is calculated by summing the 
        posterior weights across observations for each expert.}
        \item{\footnotesize \textbf{Note:} Mean and median values are applied to individuals
        in the census sample that are classified based on the value of their posterior weights.
        For example, observation $i$ is assigned to expert $j$ if the posterior vector's
        largest value is the $j$-th index: $\argmax \boldsymbol{h}_{i} = h_{ij}$}
      \end{tablenotes} \label{tbl:HME3_sample_comparison}
    \end{threeparttable}
  \end{table}


\printbibliography

\end{document}
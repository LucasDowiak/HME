\documentclass[12pt]{article}

\usepackage{amssymb}
\usepackage{amsfonts}
\usepackage{amsmath}
\usepackage{mathbbol}
%\usepackage{threeparttable}
\usepackage{geometry}
\usepackage{pdflscape}
%\usepackage{algorithmic}
%\usepackage{algorithm}
\usepackage{graphicx}

\usepackage[
  backend=biber,
  style=alphabetic,
  citestyle=chicago-authordate
]{biblatex}

\setlength{\parindent}{0pt}

\bibliography{hme_references}

\graphicspath{{/Users/lucasdowiak/hme/article/images/}}

 
\title{Econometric Applications of Hierarchical Mixture of Experts}
\author{Lucas C. Dowiak}

\begin{document}
 
\maketitle{}


Department of Economics, City University of New York\smallskip, Graduate
Center,

New York, NY, 10016, \textit{Email: ldowiak@gc.cuny.edu}

\qquad

\begin{quotation}
\textbf{Abstract:}
\end{quotation}

\vspace{1pt}

\begin{quotation}
\textbf{Keywords}: Hierarchical mixture of experts, expectation maximization

\textbf{JEL Classification}: 
\end{quotation}

\vspace{1pt}

\section{Introduction}

The concepts of mixture models and mixture distributions are old hat
in the economics business. \cite{Hamilton1989} and \cite{GoldfeldQuant1973}
are a few of the pioneering works for time series and cross sectional
regression, respectively. We are also knee deep in the age of machine
learning and it's reigning champion, the artificial neural network, has
been successfuly adapted and studied in the context of applied econometrics.
This article adds to the small body of literature that employ a specific
neural network architecture to model the weights of a mixture model. In doing
so, we leverage the highly flexible nature of a neural network but maintain
interpretability and the means to quantify marginal effects.
The model under investigation is called the Hierarchical Mixture
of Experts (HME). The literature for this particular model traces back to
\cite{JJNH1991} who proposed using a mixture of experts (ME) to model vowel
classifications. In this original work, the weights given to each expert the
result of a multinomial logistic regression. The section of the model that
partitions the input space and assigns expert probabilities is refered to as
the gating network. Extending this approach, \cite{JordanJacobs1993} reshape
the gating network into a tree structure, with a set of recursive
(hierarchical) splits.

\bigskip

\cite{WaterhouseRobinson1995} puts forth a method to grow an HME from a 
single split from the root node. The authors are influenced by the growing
technique used for classification and regression trees (\cite{CART1984}) and
apply it to an HME structure. Once the gating structure to an HME tree has
been grown, the authors put forth an additional trimming algorithm as well.
\cite{FFW1997} consider (\cite{WaterhouseRobinson1995}) and alter their
growing algorithm with a mind to scaling the model to handle thousands of
experts.

\cite{JordanXuConverge1995} An extended discussion on the convergence of the
model used by \cite{JordanJacobs1993} ? (VERIFY). The authors also suggest
algorithmic improvements to help with estimation.

\cite{JiangTanner1999} discuss convergence rates of an HME model where experts
are from the exponential family with generalized linear mean functions.

\cite{JiangTanner2000} provide regularity conditions on the HME structure for
for a mixture of general linear models estimated by maximum likelihood to
produce consistent and asymptotically normal estimates of the mean response.
The conditions are validated for poisson, gamma, gaussian, and binomial experts.


\subsection{Don't leave behind time series analysis}

\cite{WMS1995} provides a detailed discussion about examining ME applied in
a time series context and provide valuable insights to avoid overfitting the
model to the data, a common problem in neural network applications.

\cite{HuertaJiangTanner2003} Extends \cite{WMS1995} to an HME.
Five and a half decades of monthly US Industrial Production
Index data. They allowed the series to choose between two models, one modeled
as a random walk and the other as trend stationary. In addition, they present a
Bayesian approach to estimation.

\cite{CarvalhoTanner2006} refocus the discussion on MoE of time series
regressions restricted to exponential family distributions. Distilling
the available literature at the time, the authors cover the import topics of
estimation and asymptotic properties in the maximum likelihood
framework, selection of the number of experts, and model validation and
fitting.

\cite{CarvalhoSkoulakis2010} Applies mixture-of-experts of a single time series.
Using stock returns, the authors structure the gating network using lagged
dependent variabels and an 'external' covariate capturing a measure of the
trade volume at that time. THIS NEEDS A LOT MORE...Mention Simulations...READ 
THIS ONE AGAIN




\subsection{Additional Articles to Include}

A.X. Carvalho, M.A. Tanner, "Hypothesis testing in mixtures-of-experts of generalized linear time series", Computational Intelligence for Financial Engineering 2003. Proceedings. 2003 IEEE International Conference on, pp. 285-292, 2003.

A.X. Carvalho, M.A. Tanner, "Mixtures-of-experts of autoregressive time series: asymptotic normality and model specification", Neural Networks IEEE Transactions on, vol. 16, pp. 39-56, 2005, ISSN 1045-9227. 

\cite{UedaGhahramani2002}

\cite{BishopSvenson2003} find previous bayesian approaches to estimating an HME
lacking [\cite{HuertaJiangTanner2003}, \cite{UedaGhahramani2002}]. Using
variational inference, the authors provide a bayesian estimation approach
to the log marginal likelihood. With an eye to prediction, the author's advocate
that their approach makes the HME model easier to estimate without overfitting.
[Discuss how the authors approach model selection]


\cite{CarvalhoSkoulakis2005}

\bigskip

This article investigates a deviation from the original HME as proposed by
much of the previous literature. Instead of having a unique expert at the 
end of any one path from the root node, we structure our gating expert to 
allow multiple paths from the root node to a given expert. 

This allows us to fix the number of experts but still allow the gating network
to be as flexible as it needs to be. This contrasts with the original structure
in that as the complexity (define complexity?) of the gating network increases,
the number of unique experts has to increase as well.

In the context of a time series, this generalization makes it possible to
model a series that starts in one regime, temporarily switches to a second,
and then reverts back to the original. This kind of behaviour is easily captured
by a markov switching model (\cite{Hamilton1989}) but has no direct analog
when applied to the original ME (\cite{WMS1995}) and HME
(\cite{HuertaJiangTanner2003}) setup. Figure
(\ref{fig:HMEcomparison}) depicts the differece between an HME as described
in the current literature and what is proposed in this paper.

\begin{figure}[ht]
  \centering
  \includegraphics[width=\textwidth]{Org_vs_Ext.jpeg}
  \caption{Two Hierarchical Mixture of Expert models with symmetric
  binary splits and a depth of two. Gating nodes are depicted as circles
  and experts are rectangles. The root node $N^{0}$ lies at the apex and
  splits towards two additional gating nodes ($N^{1|0}$, $N^{2|0}$) one layer
  down, each providing an  additional layer of soft-partitioning through a
  binary partition of the input space. On the left is the original expert
  design, where each path flows into a unique expert. The modification
  proposed in this paper is given on the right, with the existance of multiple
  paths leading from the root node to a reduced number of experts.}
  \label{fig:HMEcomparison}
\end{figure}


\section{Model}

The model is structure as a tree that branches out in successive
layers. The non-terminal nodes of the tree are called "gating" nodes and
split the input space ($Z$) into regions. The location of these nodes will be
specified by their address $n^{a}$. The root node resides at the apex of the tree
and has the address $n^{0}$. At its most general, each gating node can yield an
arbitrary number of splits. In this paper, we restrict the number splits each
gating node can take before we estimate the model. The researcher will choose the
number of regimes to model $(M)$, and each gating node will split into $M$
directions. In figure (\ref{fig:HMEcomparison}), the gating nodes use binary
splits. The two children of the root node are $n^{1}$, which labels the
node created from the left side of the split and $n^{2}$, the node from
the right side of the split. Node $n^{1}$ will also partition the data
into a left $n^{1|1}$ and right $n^{2|1}$ nodes. This recursive splitting
continues until the last non-terminal nodes. The terminal nodes, which we will
call "experts" will be given an inded from 1 to $M$. For instance, the model on 
the left of figure (\ref{fig:HMEcomparison}) has eight experts while the model on
the right has two experts, $P^{1}$ and $P^{2}$.





\subsection{Experts}
It is helpful to think of the HME as a mixture model with the terminal
nodes of the tree in figure (\ref{fig:HMEcomparison}) representing conditional
distributions. These conditional distribution are usually formed by some flavor of
regression. For cross-sectional analysis, \cite{JordanJacobs1993} provide
a framework using generalized linear models (see \cite{GLM1972}). This type
of network is suitable for time-series regression as \cite{HuertaJiangTanner2003}
lays out in detail. For a given input and output pair $(X_{t}, Y_{t})$,
each expert provides a probablistic model relating input $X_{t}$ to output
$Y_{t}$ and can be summarized as:

\begin{equation} \label{eq:ConditionalDistribution}
  P^{m}_{t} \equiv P^{m}(Y_{t}|X_{t}, \beta^{m}), \quad m = 1,2,...,M
\end{equation}

The experts are combined into a mixture distribution denoted:

\begin{equation} \label{eq:mixture}
  P(Y_{t}|X_{t}, Z_{t}; \boldsymbol{\omega}, \boldsymbol{\beta}) = \sum_{m=1}^{M}\mathbb{\Pi}_{t}(m | Z_{t}, \boldsymbol{\omega})P^{m}(Y_{t} | X_{t},\beta^{m})
\end{equation}

where $m$ is one of the $M$ component experts in the mixture and
$\mathbb{\Pi}_{t}(m | \cdot)$ is the probability that the input unit $t$ belongs
to expert $m$. Each input unit $X_{t}$ is assigned a probability vector
$[\mathbb{\Pi}_{t}(1 | \cdot), \mathbb{\Pi}_{t}(2 | \cdot), \ldots, \mathbb{\Pi}_{t}(M | \cdot)]$
of membership to each expert with the usual restrictions:
$0 \leq \mathbb{\Pi}_{t}(m | \cdot) \leq 1$ for each $m$ and
$\sum_{m} \mathbb{\Pi}_{t}(m | \cdot) = 1$.

\subsection{Gating Network and $\mathbb{\Pi}_{t}(m | Z_{t}, \boldsymbol{\omega})$}
The main thrust of the HME architecture is the way it arrives at the mixture
weights for each input vector. The probability of each input unit $x_{t}$
belonging to expert $m$ is calculated, in recursive fashion, by traversing the
path from the root node to the the terminal gating node. The number of
recursive steps is equal to the depth of the tree and each non-terminal node
adds another layer of soft partitioning to the weights in equation 
\ref{eq:mixture}. Each node itself is a multinomial model. This paper will 
refer to an arbitrary node as $g^{a}$, and to a specific split this gate
takes as $g^{a,i}$.

\begin{equation} \label{eq:softmax}
  g^{a,i}_{t} \equiv g^{a,i}_{t}(Z_{t}, \boldsymbol{\omega}^{a}) = \frac{\exp(Z_{t}^{\prime} \ \omega^{a,i})}{\sum_{m}{\exp(Z_{t}^{\prime} \ \omega^{a,m})}}
\end{equation}

(the multinomial splits are subject to the usual identification restrictions.
See Davidson and MacKinnon pg 469)The tree structure in figure () has binary
splits, which reduces the multinomial model to a logistic model. It is important
to keep track of the product
path an input vector travels from one node to another. If the observation index is
supressed, the product path from one node (say the root node $n^{0}$) to
another (say $n^{m|\ldots|j|i}$) can be defined as

\begin{equation} \label{eq:gpath}
  \pi_{n^{0} \longleftrightarrow n^{m|\ldots|j|i}} =
    \begin{cases} 
       g^{0, i} \ g^{i|0, j} \ldots g^{\dots|j|i|0,m} & \textrm{if path is feasible} \\
       1 & \textrm{otherwise}
    \end{cases}
\end{equation}

If one of the nodes is an expert, then multiple paths may exists from the (repeated)
expert to the higher-level gating node (as in figure (\ref{fig:HMEexample})). If this is
the case, we can index these multiple paths by $l$ so that:

\begin{equation} \label{eq:pathsums}
  \mathbb{\Pi}_{t}(m | Z_{t}, \boldsymbol{\omega}) = \sum_{l}\pi_{n^{0} \overset{l}{\longleftrightarrow} P^{m}}
\end{equation}

By collecting--and summing--all possible paths from the root node to each
expert, the probability given in equation (\ref{eq:mixture}) can be expanded
and expressed as a set of conditional probabilities.

\begin{equation} \label{eq:contribution}
  \begin{split}
    P(Y_{t}|X_{t}, Z_{t}; \boldsymbol{\omega}, \boldsymbol{\beta}) =& \sum_{m} \mathbb{\Pi}_{t}(m | Z_{t}, \boldsymbol{\omega}) P^{m}(Y_{t}|X_{t},\beta^{m}) \\ 
      =& \sum_{m} P^{m}(Y_{t}|X_{t}, \beta^{m}) \sum_{l}\pi_{n^{0} \overset{l}{\longleftrightarrow} P^{m}}
  \end{split}
\end{equation}

The product of these individual probabilities across the full sample size $T$ yields
the likelihood function.

\begin{equation} \label{eq:likelihood}
  \mathcal{L}(\boldsymbol{\theta}| Y, X, Z) = \prod_{t}\sum_{m}P^{m}(Y_{t}|X_{t},\beta^{m})\sum_{l}\pi_{n^{0} \overset{l}{\longleftrightarrow} P^{m}}
\end{equation}

And taking its log yields to log likelihood

\begin{equation} \label{eq:loglikelihood}
  \boldsymbol{l}(\boldsymbol{\theta}|Y, X, Z) = \sum_{t}\log\sum_{m}P^{m}(Y_{t}|X_{t},\beta^{m})\sum_{l}\pi_{n^{0}\overset{l}{\longleftrightarrow} P^{m}}
\end{equation}

The functional form of the log likelihood (\ref{eq:loglikelihood}) does not
lend itself easily to direct optimization, but a well established
technique using expectation maximization (\cite{EM_DLR1977}) to estimate mixture
models is available. This was the primary insight of \cite{JordanJacobs1993}'s
original paper.

\section{The EM Set-Up}
The EM approach to estimating an HME model starts by suggesting that if a
resarcher had perfect information, each input vector $X_{t}$ could be matched
to the expert $P_{m}$ that generated it with certainty. If a set of indicator
variables is introduced that captures this certainty, an \textit{augmented}
version of the likelihood in equation (\ref{eq:likelihood}) can be put forward.
Define the indicator set as:

\begin{equation} \label{eq:indicator}
  I_{t}(m) = \begin{cases} 
     1 & \textrm{if input vector $X_{t}$ is generated by expert $m$} \\
     0 & \textrm{otherwise}
             \end{cases}
\end{equation}

We can then reformulate the likelihood equation

\begin{equation}  \label{eq:likelihood}
  \mathcal{L}_{c}(\boldsymbol{\theta}|Y, X, Z) = \prod_{t}\prod_{m}[P^{m}(Y_{t}|X_{t},\beta^{m})\sum_{l}\pi_{n_{0}\overset{l}{\longleftrightarrow} P^{m}}]^{I_{t}(m)}
\end{equation}

leading to the complete-data log-likelihood

\begin{equation}  \label{eq:loglikelihood}
  \boldsymbol{l}_{c}(\boldsymbol{\theta}|Y, X, Z) = \sum_{t}\sum_{m}I_{t}(m)[\log P^{m}(Y_{t}|X_{t},\beta^{m}) + \log\sum_{l}\pi_{n_{0}\overset{l}{\longleftrightarrow} P^{m}}]
\end{equation}

In previous works, each terminal node of the tree was usually its own unique
expert. This assumption would simplify equation (\ref{eq:likelihood})
by removing the summation of the gating paths to expert $m$ into a 
single path,
$\prod_{t}\prod_{m}[P^{m}(Y_{t}|X_{t}, Z_{t}, \beta^{m}) \pi_{n_{0} \longleftrightarrow P^{m}}]^{I_{t}(m)}$,
simplifying its value and the value of it's gradient and hessian. This paper
allows the analyst to specify the number of experts before optimization,
adding a bit more control over the mixture model but at the cost of a more
complicated complete likelihood function. In fact this paper will test whether
this can lead to any improvements, specifically the concers regarding overfitting
(see \cite{BishopSvenson2003} for more references on this) when an expert latches
on to a single observation, producing a singularity in the likelihood function.

\subsection{E-Step}
The E-step of the algorithm performs an expectation of over the log-likelhood
equation (\ref{eq:loglikelihood}).

\begin{equation} \label{eq:Estep}
  \begin{split}
  Q(\boldsymbol{\theta}) = \mathbb{E}\left[\boldsymbol{l}_{c}(\boldsymbol{\theta}|Y,X,Z)\right] & = \sum_{t}\sum_{m} \mathbb{E}\left[I_{t}(m)\right] [\log P^{m}(Y_{t}|X_{t},\beta^{m}) + \log\sum_{l}\pi_{n^{0}\overset{l}{\longleftrightarrow} P^{m}}] \\
   & = \sum_{t}\sum_{m} [h^{0,m}_{t} \log P^{m}(Y_{t}|X_{t},\beta^{m}) + h^{0,m}_{t} \log \sum_{l}\pi_{n^{0}\overset{l}{\longleftrightarrow} P^{m}}]
 \end{split}
\end{equation}

Here $h^{0,m}_{t} \equiv \mathbb{E}\left[I_{t}(m)\right] = \mathbb{P}[I_{t}(m) = 1| y_{t}, x_{t}, z_{t}; \boldsymbol{\theta}]$
is what we define as the posterior probability of oberserved value $y_{t}$ being
produced by expert $m$, given our tree structure starts at the root node $n^{0}$.
The posterior weight for any gating node $n^{a}$ can be calculated by

\begin{equation} \label{eq:posteriornode}
  h^{a,m}_{t} = \frac{P^{m}_{t}\sum_{l}\pi_{n^{a}_{t}\overset{l}{\longleftrightarrow} P^{m}_{t}}}{\sum_{k}P^{k}\sum_{l}\pi_{n^{a}_{t}\overset{l}{\longleftrightarrow} P^{k}_{t}}}
\end{equation}

If each expert is unique, there is only a single path from the root node to each
expert. In equation (\ref{eq:Estep}), this implies that $l = 1$ and the researcher
can avoid taking the log of a summation, greatly simplifying the value of the
gradient and hession. On the other hand, if the same expert appears at the end of
separate paths along the gating network, at some point (depending on the network's
structure) we still need to take the log of a summation. If the researcher wants
to specify the number of expert regimes beforehand, she can do so but at the
expense of a more complicated complete log-likelhood equation.



\subsection{M-Step}

Note that the paramters governing the experts and the gating network in
equation (\ref{eq:Estep}) are additively seperable. Taking each in turn, the
Jacobian of the gating portion of equation (\ref{eq:Estep}) is:

\begin{equation} \label{eq:gateJacobian}
  \begin{split}
    \frac{\partial Q}{\partial \omega^{a,i}_{p}} =& \sum_{t} \sum_{m=1}  \frac{h^{0,m}_{t}}{\sum_{l} \pi_{n^{0}_{t} \overset{l}{\longleftrightarrow} P^{m}_{t}}} \sum_{l} \frac{\partial \pi_{n^{0}_{t} \overset{l}{\longleftrightarrow} P^{m}_{t}}}{\partial \omega^{a,i}_{p}} \\
  \end{split}
\end{equation}

Where we have the ratio of the posterior $h^{0,m}$ and prior 
$\sum_{l} \pi_{n^{a} \overset{l}{\longleftrightarrow} P^{m}}$ weights with resepect
to the root node and the gradient of the posterior weight with respect to the node
in questions $\frac{\partial g^{a,m}}{\partial \omega^{a,i}_{p}}$.





\section{Standard Errors}

\begin{equation} \label{eq:gateHessian}
  \begin{split}
    \frac{\partial^{2} Q}{\partial \omega^{a,i}_{p} \partial \omega^{a',j}_{q}} =& \sum_{t} \sum_{m} \left[  \frac{h^{0,m}_{t}}{\sum_{l} \pi_{n^{0}_{t} \overset{l}{\longleftrightarrow} P^{m}_{t}}} \frac{\partial \pi_{n^{0}_{t} \overset{l}{\longleftrightarrow} P^{m}_{t}}}{\partial \omega^{a,i}_{p}} - \left( \frac{h^{0,m}_{t}}{\sum_{l} \pi_{n^{0}_{t} \overset{l}{\longleftrightarrow} P^{m}_{t}}} \right)^{2} \sum_{l} \sum_{l'} \frac{\partial \pi_{n^{0}_{t} \overset{l}{\longleftrightarrow} P^{m}_{t}}}{\partial \omega^{a,i}_{p}}  \frac{\partial \pi_{n^{0}_{t} \overset{l'}{\longleftrightarrow} P^{m}_{t}}}{\partial \omega^{a',j}_{q}} \right]
  \end{split}
\end{equation}

-- Put his directly after the log-likelihood equation in the EM section
-- Reference \cite{MengRubinSEM1991}
-- Show how weights enter into the score and hessian

\bigskip

This isn't correct

\begin{equation} \label{eq:gatinghessianQ}
  \mathbf{A}^{a} = (\mathbf{I}_{s-1} \otimes \mathbf{Z})^{\prime} \mathbf{G}^{a} (\mathbf{I_{s-1}} \otimes \mathbf{Z})
\end{equation}

\begin{equation} \label{eq:nodehessian}
  \mathbf{G}^{a} = h^{a} \begin{bmatrix}
     g^{1|a}(1-g^{1|a}) & -g^{1|a}g^{2|a}    & \dots  & g^{1|a}g^{s-1|a}       \\
     -g^{1|a}g^{2|a}    & g^{2|a}(1-g^{1|a}) & \dots  & g^{2|a}g^{s-1|a}       \\
     \vdots             &                    & \ddots & \vdots                 \\
     g^{1|a}g^{s-1|a}   &  \dots             &        & g^{s-1|a}(1-g^{s-1|a}) \\
  \end{bmatrix}
\end{equation}

For multinomial regressions, each 



\section{Average Gating Marginal Effect}
There is an issue of how to tie the variables in the gating network to the regression relationship described by experts. For logit/probit regression, there is the standard margins at the means (MEM) and the average marginal effect (AME). We should be able to extend this kind of analysis to the entire weighting network, chaining marginal effects from the root node down and determind/quantify/summarize the effect each gating factor has on directing which observation to which expert. These can then be tied with the marginal regression effect. Starting with \ref{eq:gpath}, and summing over all the paths from the root node to expert $m$, we can re-arrange the products of the collected path as follows:  

\begin{equation}
  \begin{split}
  w^{m} &= \sum_{l} \pi_{n_{0} \overset{l}{\longleftrightarrow} P^{m}} \\
        &= \sum_{i} \sum_{j} \cdots \sum_{k} g^{i} g^{j|i} \cdots g^{m|k|\cdots|j|i} \\
%  \frac{\partial w^{m}}{\partial Z_{k}} &= \frac{\partial g^{1}}{\partial Z_{k}}g^{1|1} + g^{1}\frac{\partial g^{1|1}}{\partial Z_{k}} + \frac{\partial g^{2}}{\partial Z_{k}}g^{1|2} + g^{2}\frac{\partial g^{2|2}}{\partial Z_{k}} \\ 
  \end{split}
\end{equation}

Applying the chain, we can calculate the marginal effect that variable $p$ has on the weight assinged to each observation being assinged to expert $m$.

\begin{equation}
  \begin{split}
    \frac{\partial w^{m}}{\partial Z_{p}} &= \sum_{i} \frac{\partial g^{i}}{\partial Z_{p}} \sum_{j} g^{j|i} \cdots \sum_{k} g^{m|k|\cdots|j|i} \\
    &+ \sum_{i} g^{i} \sum_{j} \frac{\partial g^{j|i}}{\partial Z_{p}} \cdots \sum_{k} g^{m|k|\cdots|j|i} \\
    &+ \sum_{i} g^{i} \sum_{j} g^{j|i} \cdots \sum_{k} \frac{\partial g^{m|k|\cdots|j|i}}{\partial Z_{p}} \\
  \end{split}
\end{equation}



\section{Miscellaneous}



\subsection{Model Specification}
When implementng a HME model, one of the major decision points is the exact specification for each of the model's experts. If comparing a trend-stationary model versus a random walk with drift (see \cite{HuertaJiangTanner2003}), the correct specification for each expert requires the inclusion of the true number of lagged dependent variables. While standard approaches such as the autocorrelation and partial autocorrelation functions exist, the multi-state nature of model makes this appreach difficult due to the uncertianty surrounding which state is governing the series at any given time $t$. Unless the researcher has some previous knowledge of the model's form, they are left with performing standard model selection for each expert on the entire series or some subset of the series. To overcome this issue, we follow simple procedure in this paper.

\begin{enumerate}
\item The HME model is run where each expert is an AR(1) model without a constant.
\item Use AIC/BIC to perform model selection on each expert, performed on the full series, where the each contribution to the likelihood has been weighted using the posterior weights from equation (\ref{eq:posterior}).
\item Perform a final HME run using the specifications obtained in step (2) for each expert.
\end{enumerate}

The validity of this approach is tested on two sets of simulated autoregressive data. this approach

% \begin{table}
% \caption{Model Selection}{c}
% \begin{threeparttable}
%  \begin{tabular}[l]{l c c c | c c c || l c c c | c c c}
%    \hline \hline
% \multicolumn{7}{c}{Simulated HME-AR$^{1}$} & \multicolumn{7}{c}{Simulated Univariate AR} \\
% & \multicolumn{3}{c}{AIC$^{2}$} & \multicolumn{3}{c}{BIC$^{3}$} & & \multicolumn{3}{c}{AIC$^{2}$} & \multicolumn{3}{c}{BIC$^{3}$} \\
% N & E$_{1}$ & E$_{2}$ & E$_{3}$ & E$_{1}$ & E$_{2}$ & E$_{3}$ & N & E$_{1}$ & E$_{2}$ & E$_{3}$ & E$_{1}$ & E$_{2}$ & E$_{3}$ \\
% \hline
% 100   & 0.29 & 0.09 & 0.30 & 0.66 & 0.23 & 0.61 & 33  & 0.67 & 0.32 & 0.62 & 0.9 & 0.29 & 0.76 \\
% 200   & 0.18 & 0.25 & 0.29 & 0.71 & 0.60 & 0.58 & 66  & 0.68 & 0.62 & 0.72 & 0.98 & 0.43 & 0.95 \\
% 300   & 0.25 & 0.39 & 0.27 & 0.69 & 0.73 & 0.63 & 100 & 0.65 & 0.77 & 0.77 & 0.95 & 0.72 & 0.92 \\
% 400   & 0.19 & 0.34 & 0.18 & 0.70 & 0.83 & 0.73 & 133 & 0.70 & 0.66 & 0.73 & 0.99 & 0.77 & 0.98 \\
% 500   & 0.25 & 0.33 & 0.27 & 0.76 & 0.88 & 0.73 & 166 & 0.66 & 0.69 & 0.69 & 0.98 & 0.92 & 0.97 \\
% 600   & 0.19 & 0.33 & 0.17 & 0.74 & 0.86 & 0.72 & 200 & 0.78 & 0.74 & 0.75 & 0.99 & 0.94 & 0.96 \\
% 700   & 0.19 & 0.25 & 0.12 & 0.73 & 0.86 & 0.69 & 233 & 0.73 & 0.73 & 0.69 & 0.97 & 0.96 & 0.97 \\
% 800   & 0.17 & 0.28 & 0.21 & 0.82 & 0.91 & 0.75 & 266 & 0.68 & 0.75 & 0.77 & 0.99 & 1.00 & 0.95 \\
% 900   & 0.22 & 0.30 & 0.24 & 0.77 & 0.83 & 0.72 & 300 & 0.74 & 0.77 & 0.76 & 0.97 & 0.99 & 0.98 \\
% 1000  & 0.14 & 0.34 & 0.29 & 0.80 & 0.90 & 0.70 & 333 & 0.69 & 0.76 & 0.77 & 0.98 & 0.98 & 0.99 \\
% \hline \hline
%   \end{tabular}
%   \begin{tablenotes}
%     \item Values in the tables are percentages, out of 100 simulations, of obtaining the correct lag order of the autoregressive process for each expert. The size of the simulated series is given by first and eigth columns (N). The experts in the HME share the same specification as their univariate counterparts.
%     \item[1] Expert AR(p) specification:
%     Expert 1: ($\phi_{1}=0.7$, $\sigma^{2}=0.03$);
%     Expert 2: ($\phi_{1}=-0.5$, $\phi_{2}=0.3$, $\sigma^{2}=0.05$);
%     Expert 3: ($\phi_{1}=0.05$, $\phi_{2}=0.1$, $\phi_{3}=0.6$, $\sigma^{2}=0.1$)
%     \item[2]{AIC: -2 $\times$ loglikelihood + 2 $\times$ p}
%     \item[3]{BIC: -2 $\times$ loglikelihood + ln(N) $\times$ p}
%   \end{tablenotes}
% \label{tbl:modelselection}
% \end{threeparttable}
% \end{table}

\begin{enumerate}
\item Future research: Agostinelli and Markatou \cite{AgostinelliMarkatou} investigated weighted likelihood ratio tests
\end{enumerate}

For a time series that moves back and forth between separate regiems, including a time trend can be tricky. Even though the weighting scheme allows different sets of observations to be more important than others when estimating model parameters, once the model parameters have been estimated, they are prescriptive for the entire series. Basically, if one model estimates a trend parameter, the trend for that entire expert is now set, and if/when the time series exists that particular regime and re-enters at a later date, it is re-entering at a level described by the trend.

\subsection{Wald Test is Invalid}
S+plus GLM section on problems with binomial GLMs
  -- Hauck and Donner (1977) JASA \cite{HauckDonner} Quoting S+plus: If there are some $\hat{\beta}_{i}$ that are large, the curvature of the log-likelihood at $mathbf{\hat{\beta}}$ can be much less than near $\beta_{i}=0$, and so the Walk approximation understimates the change in log-likelihood on setting $\beta_{i}=0$. This happens in such a way that as $|\hat{\beta}_{i}| \rightarrow \infty$. Thus highly significant coefficients according to the likelihood ratio test may have non-significant t ratios ............. There is one fairly common circumstance in which both convergence problems and the Hauek-Donner phenomenon can occure. This is when the fitted probs are extremely close to zero or one.



\subsection{Simulations}
In order to carry out simulated experiments of the HME framework, this paper will follow the following methodology.

First, we decide on the model specification $M^{i}(\phi)$ for each expert (AR(p), trend stationary, random walk, etc.), including the choose error distribution. These models are then parameterzied with given values $\phi=(\phi^{1},\dots,\phi^{M})$ Parameters are choosen for each model

Second, we a assign prior probabilties to each expert that summarizes the expectation of each expert being responsible for any given input vector $Z_{t}$. In most cases, when time is the only variable in the gating network, the transition between experts will be a smooth function of time such as the logistic function or a rescaled hyperbolioc tangent. In this paper, prior probabilities for two and three expert models are smooth transitions structured in a way that each expert governs the series for roughly the same amount of time (figure \ref{fig:expertmembership} provides a graphical example). We can organize our prior weights for each expert into a matrix, $G$, whose columns represent separate experts and rows index input patterns. For notational purposes, the M-vector of proabibility of belonging to each expert for input vector $t$ is written as $G_{t}$, while the T-vector of expert $i$'s membership across all input patterns will be written as $G^{s=i}$.

\begin{figure}[ht]
  \centering
  \includegraphics[width=\textwidth]{images/smooth_expert_membership.jpeg}
  \caption{Example of the smooth transitions probabilities between experts used in this study's simulations.}
  \label{fig:expertmembership}
\end{figure}

This method of simulation, with a small tweak, provides a route to obtaining bootstrapped confidenced intervals for the estimated parameters of the an HME model. The only change is replace the random draws from a fixed distribution $F$ Another approach to parameter inference is the venerable bootstrap. If our gating network is a function of time, the following strategy using bootstrapped residuals can be used to find confidence intervals for our parameter estimates. Once an HME model is estimated, we can organize our prior weights for each expert into a matrix, $G$, where each column $s$ represents the (prior) probability of expert . Note that each row sums to one. Posterior expert weights can be collect in a similar matrix called $H$.

% \begin{algorithm}
%   \caption{Bootstrap Sample}
%   \begin{algorithmic}[1]
%     \REQUIRE $H$, $G$, $M(\hat{\phi})$, $\hat{e}$, $T$, $T_{B}$
%     \FOR{$t=-T_{B}$ to $T$}
%       \IF{$t < 1$}
%         \STATE $s \thicksim G_{t=1}$
%       \ELSE
%         \STATE $s \thicksim G_{t}$
%       \ENDIF
%       \STATE $\epsilon^{*}_{t} \thicksim \hat{e}^{i=s}$ with weights $H^{i=s}$
%       \STATE $y^{*}_{t} \leftarrow M^{i=s}(y^{*}_{-T_{B},...,t-1},\hat{\beta}^{i=s}) + \epsilon^{*}_{t}$
%     \ENDFOR
%     \RETURN $y^{*}_{1,...,T}$
%   \end{algorithmic}
% \end{algorithm}

\section{Diagnostics}
pg 7 of \cite{WMS1995} suggested observing the distribution of the terminal
$g_{i}$. If only one expert is responsible for each observations, $g_{i}$ will
be close to one for a single expert and near zero for all other experts.
Can we formalize this comparison in to a specific test?

density forecasts evaluations.

Standard likelihood-ratio test is not valid (\cite{CarvalhoTanner2006}) with
AIC/BIC/VOUNG test being prefered.


  
\section{Discusion}
\subsection{Wide networks vs Deep Networkds}
In this form, a network of a single depth is similar/equal(?) to multinomial logistic regression.
\subsection{Compare to Alternatives}
\begin{enumerate}
\item Smooth transition autogressive (STAR(p)) model has similiar logistic function but the lacks the flexibility of changing model types
\item Fixed Memory Markov Process. Shares ability to switch between different expert types. Extensions have been provided by Diebold, Lee, and Weinback \cite{DieboldLeeWeinbach1994} as well as Filardo \cite{Filardo1994} to include conditional transitional dynamics based on other covariates.
\item Smooth domain partitioning contrasts
\end{enumerate}
\begin{figure}[ht]
  \centering
  \includegraphics[width=\textwidth]{images/bias_2RD1.jpeg}
  \caption{\textbf{HME: Experts=2, Depth=1} Plot of parameter bias of a two-expert HME model as the size of the simulated series increases. Bias is defined as $b=n^{-1}\textstyle{\sum}(\beta_{i}-\beta^{*})$. Simulations were carried out as described in section \ref{standarderrors} with two separate AR(1) processes and 500 replicates per series length. Regime one is generated with parameters ($\phi_{1}=0.8$, $\sigma^{2}_{1}=0.15$). Regime two is generated with parameters ($\phi_{1}=0.5$, $\sigma^{2}_{1}=0.07$).}
  \label{fig:bias_2RD1}
\end{figure}

\begin{figure}[ht]
  \centering
  \includegraphics[width=\textwidth]{images/bias_2RD2.jpeg}
  \caption{\textbf{HME: Experts=2, Depth=2} Plot of parameter bias of a two-expert HME model as the size of the simulated series increases. Bias is defined as $b=n^{-1}\textstyle{\sum}(\beta_{i}-\beta^{*})$. Simulations were carried out as described in section \ref{standarderrors} with two separate AR(1) processes and 500 replicates per series length. A smooth probability transition is generated by a rescaled hyperbolic tanget function $\pi=\frac{\tanh(\delta x) + 1}{2}$. AR(1) parameters for regime one: $\phi_{1}=0.8$, $\sigma^{2}_{1}=0.1$. AR(1) parameters for regime two: $\phi_{1}=0.5$, $\sigma^{2}_{1}=0.03$.}
  \label{fig:bias_2RD2}
\end{figure}

\printbibliography


\end{document}

